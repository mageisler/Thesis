\chapter{Introduction \label{sec:Intro}}

The current assumption of the nature of elementary particles and their interactions is described by a handful of principles. Most of these principles compose the Standard Model of Particle Physics (SM). In it the particles are grouped by their intrinsic angular momentum called spin. All particles with half-integer spin values are named Fermions and can be associated with matter. These are then further categorized by their participation in the strong force into leptons and quarks. Up to now, the latter have not been seen individually but in bounded states like mesons (two quarks) or baryons (three quarks). All particles with an integer spin value are named bosons. They can be either elementary or composite particles like the mentioned Mesons. Elementary bosons are mostly force carriers, named gauge bosons. These are the propagators of the strong force, the eight gluons and those of the electro-weak force, namely $\gamma{}$, \Zz{} and \Wpm{}. Recently, a long searched boson has been detected, the Higgs boson. The connected Englert-Brout-Higgs field is believed to give mass to all elementary particles. \\
Mathematically, the unification of the electromagnetic and weak interactions can be described by a $\textrm{SU(2)}_{\textrm{L}} \times \textrm{U(1)}_{\textrm{Y}}$ group by the Glashow-Salam-Weinberg theory. The strong interaction can be described by quantum chromodynamics in a $\textrm{SU(3)}_{\textrm{C}}$ group. An overview of the particles is shown in Figure~\ref{plot:IntroParticles}.

\begin{figure}[!htb]
  \centering
  \includegraphics[width=0.55\textwidth]{Intro/ParticleListing}
  \caption[Overview of the elementary particles]{An overview of the elementary particles of the Standard Model. The particles are categorized into Fermions and Bosons. For each particle the mass in \MeV (upper left corner), the electromagnetic charge in multiples of the positron charge e (bottom left corner) and the spin (bottom right corner) is listed. The fermions are further split into quarks and leptons. For the neutrinos the upper bounds of the masses are given. The particles in dark colored boxed participate in the strong force. All values are taken from~\citen{Beringer:1900zz}. \label{plot:IntroParticles} }
\end{figure}

Describing only about $5\,\%$ of the actual composition of the universe, the Standard Model solves a lot of issues while some questions are still open. These include the actual number of lepton families, the hierarchy problem and the grand unification of all forces. To address the last two issues a theoretical expansion of the Standard Model has been created. In the scenario of a Supersymmetric Standard Model (SUSY) every SM-particle would obtain a supersymmetric partner with different spin. In theory, the lightest of these new supersymmetric particles could be a candidate for Dark Matter. This could open the door to about $25\,\%$ of the composition of the universe. But, up to now no sign or hint for this theory has been observed. A review of the current state o fthis theory can be found in~\cite{Beringer:1900zz}. \\

Nowadays, lots of different experiments try to test the predictions of the Standard Model as well as of additional theories like SUSY at different frontiers. One of the largest and best-known experiments is located at CERN  close to Geneva in Switzerland. The particle collider Large Hadron Collider (LHC) located there is currently the world's largest accelerator reaching also the highest energies. Together with the four main detectors it started operation at the end of 2010 and is currently (2014) in a long shutdown. After this shutdown the energy of the accelerated nuclei will be increased so that new physics is in reach and the mentioned SUSY models can be more investigated. One advantage of accelearting hadrons is the relatively small energy loss because of synchrotron radiation. On the other hand, one disadvantage is the huge amount of soft interactions, which produces lots of background. \\

In Chapter~\ref{sec:LHCCMS} more information about the properties of the LHC as well as of one of the main detectors, the Compact Muon Solenoid (CMS), is presented. The studies presented in this work are based on the framework of this experiment. Hence, a brief description about the structure of the detector and the event reconstruction is shown. Furthermore, some key facts about the actual number of proton-proton interactions in one bunch crossing. Connected with this number are the unavoidable background interactions and their influence on the reconstruction. \\
A new approach to reduce this influence is presented in this scope. A detailed description of the work flow of the association of tracks to the signal and background interactions is given in Chapter~\ref{sec:AssMap}. Starting with a comparison of the performance of the track association of this new approach and two other already existing approaches presented in Chapter~\ref{sec:TrackAss}. Following, also the effects on high-level objects is discussed. In Chapter~\ref{sec:OO} this is shown for the relative isolation, the reconstruction of jets, the missing transverse energy and the efficiency in b tagging. \\
In Chapter~\ref{sec:OSP} the step from tracks, which can only be created for charged particles, to all particles is shown. Furthermore, a comparison of the time and memory consumption of the different background subtraction techniques is given. \\
All studies presented earlier are based on simulations. Hence, in Chapter~\ref{sec:DC} a comparison is shown between data and simulation for some observables like the number of tracks that are considered as signal tracks. Additionally, the results of the corrected missing transverse energy for the different pileup subtraction techniques is investigated. \\
In the last chapter a small outlook to the performance of pileup subtraction is drawn based on simulation of the upcoming run conditions. As shown in Chapter~\ref{sec:HPU}, one problem that needs to be faced in the future is the high density of the primary vertices in the interaction region. Under these conditions a good association of tracks to primary vertices is challenging.

\paragraph*{Natural Units}

In this thesis natural units are used when beneficial. These are commonly used in high energy physics, equalizing $\hbar{} = c = 1$. This has as a consequence that energy, momentum and masses have the same dimension. SI units are used only if explicitly denoted (for instance, to keep clearness for distances).
