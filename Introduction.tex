\chapter{Introduction \label{sec:Intro}}

The current knowledge of the nature of elementary particles and their interactions can be described by a handful of principles constituting the Standard Model of Particle Physics (SM). In it the particles are grouped by their intrinsic angular momentum called spin. All particles with half-integer spin values are named Fermions and can be associated with matter due to Fermi-Dirac statistics. These are then further categorized by their participation in the strong force into quarks and leptons. Up to now, the quarks have not been seen bare but in bounded color-neutral states like mesons (two quarks) or baryons (three quarks). All particles with an integer spin value are named bosons. They can be either elementary or composite particles like the mentioned Mesons. Elementary bosons are the propagators of the strong force, the eight gluons and those of the electro-weak force, namely $\gamma{}$, \Zz{} and \Wpm{}. Recently, a long searched boson has been detected, the Higgs boson~\cite{Chatrchyan201230}. The connected Englert-Brout-Higgs field is believed to give mass to all elementary particles. An overview of the particles is shown in Figure~\ref{plot:IntroParticles}.

\begin{figure}[!htb]
  \centering
  \includegraphics[width=0.55\textwidth]{Intro/ParticleListing}
  \caption[Overview of the elementary particles]{An overview of the elementary particles in the Standard Model. The particles are categorized into Fermions and Bosons. For each particle the mass in \MeV (upper left corner), the electromagnetic charge in multiples of the positron charge e (bottom left corner) and the spin (bottom right corner) is listed. The fermions are classified into quarks and leptons. For the neutrinos the upper bounds of the masses are given. The particles in dark colored boxed participate in the strong force. All values are taken from~\citen{Beringer:1900zz}. \label{plot:IntroParticles} }
\end{figure}

Describing only about $5\,\%$ of the actual composition of the universe, the Standard Model still leaves some questions open. For instance, the actual number of lepton families, the hierarchy problem and the grand unification of all forces. A theoretical expansion of the Standard Model has been created to address the last two issues. In the scenario of a Supersymmetric Standard Model (SUSY) every SM-Fermion would obtain a new supersymmetric bosonic partner and vice versa. But, up to now no sign for this theory has been observed. A review of the current state of this theory can be found in~\cite{Beringer:1900zz}.

Nowadays, lots of different experiments try to test the predictions of the Standard Model as well as of additional theories like SUSY at different frontiers. One of the largest experiments is located at CERN close to Geneva in Switzerland. The Large Hadron Collider (LHC) is currently the world's most powerful accelerator. Together with the four main detectors it started operation at the end of 2010 and is currently (2014) in a long shutdown phase. After this shutdown the energy of the accelerated nuclei will be increased so that new physics is in reach and the mentioned SUSY models can be investigated with more precision. One advantage of accelerating hadrons is the relatively small energy loss because of synchrotron radiation compared to electrons. On the other hand, one disadvantage is the huge amount of soft interactions, which produce lots of background processes.

In Chapter~\ref{sec:LHCCMS} information about the properties of the LHC as well as of one of the main detectors, the Compact Muon Solenoid (CMS), are presented. The studies presented in this work are based on the framework of this experiment. Hence, a brief description about the structure of the detector and the event reconstruction is shown. Furthermore, some key facts about the actual number of proton-proton interactions in one bunch crossing are given. The unavoidable background interactions and their influence on the reconstruction are connected with this number. \\
In the main part of this work a new approach to reduce this influence is presented. The key aspect of this approach is the individual association of tracks to the signal and background interactions. In Chapter~\ref{sec:AssMap} the work flow of this association is described in detail. The outcome of this sequence is a mapping between tracks and primary vertices. \\
In the following the performance of the signal and pileup identification of the new approach is compared to two established approaches. In Chapter~\ref{sec:TrackAss} this is done in terms of the track assignment based on several simulated data samples. In Chapter~\ref{sec:OSP} the transfer from tracks to all particles is explained. Moreover, a comparison of the time and memory consumption of the different background subtraction techniques is given. \\
Based on the collection of signal particles, high-level objects like jets are created. In Chapter~\ref{sec:OO} the effects of the different approaches are discussed for several observables. In this part the outcome of no pileup subtraction is also illustrated. \\
All studies presented up to this point are based on simulations. Hence, in Chapter~\ref{sec:DC}, a comparison is shown between data and simulation for some relevant observables like the number of tracks that are considered as signal tracks. Additionally, the results of the corrected missing transverse energy for the different pileup subtraction techniques is investigated. \\
In Chapter~\ref{sec:HPU} an outlook on the performance of pileup subtraction is drawn based on simulations of the upcoming run conditions. One problem that needs to be faced in the future is the high density of the primary vertices in the interaction region. Under these conditions a good association of tracks to primary vertices is challenging.

\paragraph*{Natural Units}

In this thesis natural units are used when beneficial. These are commonly used in high energy physics, equalizing $\hbar{} = c = 1$. This has as a consequence that energy, momentum and masses have the same dimension. SI units are used only if explicitly denoted (for instance, to keep clearness for distances).
