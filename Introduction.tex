\chapter{Introduction \label{sec:Intro}}

The current assumption of the nature of elementary particles and their interactions is described by a handful of principles. These principles design the Standard Model of Particle Physics (SM). In it the particles are group by their angular momentum called spin. All particles with half-integer values are named Fermions and can be associated with matter. These are then further categorized by their participation to the strong force into leptons and quarks. Up to now, the latter have not been seen individual but in bounded states like Mesons (two quarks) or Baryon (three quarks). All particles with an integer spin value are named Bosons. They can be either elementary or composite particles like the mentioned Mesons. Elementary bosons are mostly force carrier. As there are the propagator of the strong force, the gluons and those of the electro-weak-force, namely $\gamma{}$, \Zz{} and \Wpm{}. Recently, a long searched Boson has been detected, the Higgs-Boson. The purpose of the connected Higgs-field is to give mass to all elementary particles. Mathematically, the unification of the electromagnetic and weak interaction can be described by a $\textrm{SU(2)}_{\textrm{L}} \times \textrm{U(1)}_{\textrm{Y}}$-group by the Glashow-Salam-Weinberg theory. The strong interaction can be described by the Quantumchromodynamic in a $\textrm{SU(3)}_{\textrm{C}}$-group. An overview of the particles is shown in Figure~\ref{plot:IntroParticles}.

\begin{figure}[!htb]
  \centering
  \includegraphics[width=0.55\textwidth]{Intro/ParticleListing}
  \caption[Overview over the elementary particles]{An overview of the elementary particles of the Standard Model. The particles are categorized into Fermions and Bosons. For each particle the mass in \MeVcc (upper left corner), the electromagnetic charge in multiples of the positron charge e (bottom left corner) and the spin (bottom right corner) is listed. The fermions are further split into quarks and leptons. For the neutrinos the upper bounds of the masses are given. The particles in dark colored boxed participate in the strong force. All valus are taken from~\citen{Beringer:1900zz}. \label{plot:IntroParticles} }
\end{figure}

Describing only about $5\,\%$ of the actual composition of the universe the Standard Model as it now gives a lot of answers while some questions are still open. For instance, about the actual number of lepton families, the hierarchy problem and the grand unification of all forces. To address the last two issues a theoretical expansion of the Standard Model has been created. In the scenario of a Minimal Supersymmetric Standard Model (MSSM) every SM-particle would obtain a supersymmetric partner with different spin. In theory, the lightest of these new supersymmetric particles could be a candidate for Dark Matter. This could open the door to about $25\,\%$ of the composition of the universe. But, up to now no sign or hint for this theory has been observed. \\

Nowadays, lots of different experiments try to test the predictions of the Standard Model at different frontiers. One of the largest and well-known experiment is located at CERN  close to Geneva in Switzerland. The particle collider Large Hadron Collider (LHC) located at that place is currently the world largest accelerator reaching also the highest energies. Together with the four main detectors it started operation at the end of 2010 and is currently in a long shutdown. After this shutdown the energy of the accelerated nuclei will be higher so that new physics is in reach and the mentioned MSSM can be more investigated. \\

In Chapter~\ref{sec:LHCCMS} more information about the properties of the LHC as well as of one of the main detectors, the Compact Muon Solenoid (CMS), will be presented. The presented work is based on the framework of this experiment. Hence, a brief description about the structure of the detector and the event reconstruction is shown. Furthermore, some key facts about the actual number of proton-proton interactions in one bunch crossing are discussed. Connected with this number are the unavoidable background interactions and their influence on the reconstruction. \\
To reduce this influence a new approach of subtraction is presented in this scope. A detailed description of the work flow of the association of tracks to the signal and background interactions will be given in Chapter~\ref{sec:AssMap}. Starting with a comparison of the performance of the track association of this new approach and two other approaches, which already exist, will be presented in Chapter~\ref{sec:TrackAss}. Following, also the effects on high-level object will be discussed. In Chapter~\ref{sec:OO} it is done for the relative isolation, the reconstruction of jets, the missing transverse energy and the efficiency in b tagging. \\
In Chapter~\ref{sec:OSP} the step from tracks, which can only be created for charged particles, to all particles will be shown. Furthermore, a comparison of the time and memory consumption of the different pileup subtraction techniques will be given. \\
All studies presented earlier are based on simulations. Hence, in Chapter~\ref{sec:DC} a comparison will be shown between data and simulation for some observables like the number of tracks that are considered as signal tracks. Furthermore, the results of the corrected missing transverse energy for the different pileup subtraction techniques will be investigated. \\
In the last chapter a small outlook to the performance of pileup subtraction will be drawn based on simulation of the upcoming run conditions. As shown in Chapter~\ref{sec:HPU}, one problem that needs to be faced in the future is the high density of primary vertex in the interaction region. Under these conditions a good association of tracks to primary vertices is hardly possible.

\paragraph*{Natural Units}

In the work presented in this thesis natural units are partially used. These is commonly used in high energy physics equalizing $\hbar{} = c = 1$. This has as a consequence that energy, momentum and masses have the same dimension. Only if explicitly denoted SI units are used (for instance, to keep clearness at distances).
