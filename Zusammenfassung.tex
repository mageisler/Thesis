\selectlanguage{german}

\phantomsection\addcontentsline{toc}{chapter}{Kurzfassung}
\chapter*{Kurzfassung}
\markboth{Kurzfassung}{Kurzfassung}

Die vorliegende Arbeit konzentriert sich auf die Spurrekonstruktion des CMS Experimentes am LHC. Bereits am Ende der ersten Phase der Datennahme im Dezember 2012 wurde  eine instantane Luminosit\"a{}t von fast 1\ten{34}\percms{} erreicht, was im Durchschnitt etwa 20 Proton-Proton-Wechselwirkungen bei einer Kollision von zwei Protonpaketen entspricht. Dies bedeutet, dass zus\"a{}tzlich zu einer m\"o{}glichen physikalisch interessanten Interaktion viele weiche, uninteressante Wechselwirkungen stattfinden. Die Verbesserungen, die zur Zeit (2014) am Beschleuniger durchgef\"u{}hrt werden, lassen erwarten, dass die Zahl der Proton-Proton-Wechselwirkungen bei einer Kollision zweier Protonpakate weiter steigen wird. Hinzu kommt eine erh\"o{}hte Schwerpunktsenergie von bis zu 14\TeV{}, wodurch mehr Spuren pro Vertex erwartet werden. Dieser Untergrund kann dazu f\"u{}hren, dass die Aufl\"o{}sung der rekonstruierten physikalischen Objekte bei den interessanten Kollisionen verschlechtert wird. \\
Zur Zeit gibt es beim CMS Experiment zwei \"a{}hnliche Ans\"a{}tze, um die Teilchenspuren, die nicht von der interessanten Interaktion kommen, wieder aus der Kollektion aller rekonstruierten Teilchenspuren zu entfernen. Diese Arbeit stellt einen neuen Ansatz vor, der basierend auf mehr Informationen des Spurdetektors eine bessere Zuordnung erreicht. Dabei wurde die M\"o{}glichkeit implementiert, einer Teilchenspur mehrere Vertices zuzuordnen. Die Analysen konzentrieren sich auf Signalteilchenspuren, die der h\"a{}rtesten Kollision zugeordnet wurden. Durch die verschiedenen Optionen des neuen Ansatzes kann deren Leistung optimiert werden. Dies f\"u{}hrt zu besseren Ergebnissen im Vergleich zu den beiden etablierten Ans\"a{}tzen. \\
Die Auswirkungen auf die folgende Rekonstruktion von physikalischen Objekten wie zum Beispiel der Kalibrierung der fehlenden transversalen Energie wird getestet. In ann\"a{}hernd allen F\"a{}llen wird eine Verbesserung der physikalischen Ergebnisse mit dem vorgestellten Ansatz beobachtet. \\
Des weiteren werden die Ergebnisse der Simulationen mit echten Daten des CMS-Experiments verglichen. Bei die meisten Observablen zeigt sich eine gute \"U{}berstimmung. Ebenso liefert auch hier der neue Ansatz eine signifikante Verbessserung bei der Kalibrierung der fehlenden transversalen Energie. Insbesondere scheint der neue Ansatz in Szenarien mit einer h\"o{}heren Anzahl von Proton-Proton-Interaktionen zu einer signifikante Verbesserung hinsichtlich dieser Kalibrierung zu f\"u{}hren. Eine gute Zuordnung von Teilchenspuren zu Vertizes wird erschwert nach den zuk\"u{}nftigen Verbesserungen des Beschleunigers.

\selectlanguage{english}
