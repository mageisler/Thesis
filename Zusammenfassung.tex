\selectlanguage{german}

\phantomsection\addcontentsline{toc}{chapter}{Kurzfassung}
\chapter*{Kurzfassung}
\markboth{Kurzfassung}{Kurzfassung}

Die hier vorgestellte Arbeit wurde im Kontext des CMS Experimentes am LHC angefertigt. Dieser Teilchenbeschleuniger erreichte schon zum Ende der ersten Phase der Laufzeit im Dezember 2012 eine instantane Luminosit\"a{}t, welche nah an die Design-Luminosit\"a{}t von 1\ten{34}\percms{} heran reichte. Dies hatte unter anderem zur Folge, dass es bei einer Kollision von zwei Protonpaketen im Mittel zu 20 Proton-Proton Interaktionen kam. Das bedeutet, dass es neben einer m\"o{}glichen harten, physikalisch interessanten Kollision (Signal) viele weiche, uninteressante Interaktionen gab. Die Verbesserungen, die zur Zeit (2014) am Beschleuniger durchgef\"u{}hrt werden, werden diese Anzahl der Kollisionen noch weiter vergr\"o{}\ss{}ern. Hinzu kommt die erh\"o{}hte Schwerpunktsenergie von bis zu 14\TeV{}, welche die Anzahl der erzeugten Spuren pro Vertex vergr\"o{}\ss{}ert. Dieser Untergrund (auch Pileup genannt) kann dazu f\"u{}hren, dass die Aufl\"o{}sung der rekonstruierten physikalischen Objekte der interessanten Kollisionen verschlechtert wird. \\
Zur Zeit gibt es beim CMS Experiment zwei \"a{}hnliche Ans\"a{}tze, deren Ziel es ist, die Teilchenspuren, die nicht von der Signal-Interaktion kommen, wieder aus der Kollektion zu entfernen. Die angefertigte Arbeit stellt einen neuen Ansatz vor, der versucht auf Basis von mehr Informationen des Spurdetektors eine bessere Zuordnung zu erreichen. Dabei ist es m\"o{}glich, eine Teilchenspur mehr als nur einem Vertex zuzuordnen. Zur Analyse der Signalteilchenspuren werden jene Teilchenspuren verwendet, die der h\"a{}rtesten Kollision zugeordnet wurden. Durch die verschiedenen Optionen des neuen Ansatzes kann bei der Zuordnung der Signalteilchenspuren sowohl eine h\"o{}here Effizienz als auch eine bessere Kombination von Effizienz und Reinheit erreicht werden im Vergleich zu den zwei bestehen Ans\"a{}tzen. \\
Zur Analyse des Einflusses auf folgende Rekonstruktionen wurden die Ergebnisse der relativen Isolation, der \textit{Jet} Rekonstruktion, der Kalibrierung der fehlenden transversalen Energie und des Erkennens von so genannten \textit{b-Jets} studiert. W\"a{}hrend bei der Kalibrierung der fehlenden transversalen Energie der neue Ansatz die besten Ergebnisse zeigt, ist eine einfache Aussage bei der Rekonstruktion der \textit{Jets} nicht m\"o{}glich. Vor einer Korrektur des Impulses der \textit{Jets} zeigt der neue Ansatz eine geringere Abh\"a{}ngigkeit von den meisten Parametern, im Vergleich aber zu einem bestehenden Ansatz eine zu kleine Rekonstruktion des Impulses. Mit mehreren Assoziieren pro Teilchenspur kann diese Differenz verringert werden. \"A{}hnlich ist es bei dem Erkennen von \textit{b-Jets}. Mit mehreren Assoziierungen ist der neue Ansatz mindestens so Effizient wie die bisherigen Ans\"a{}tze.\\
Des weiteren wurden die Ergebnisse der Simulationen, wenn m\"o{}glich, mit echten Daten des CMS Experimentes verglichen. Bei den meisten Observablen zeigt sich eine gute \"U{}berstimmung. Ebenso liefert auch hier der neue Ansatz die besten Ergebnisse bei der Kalibrierung der fehlenden transversalen Energie. Das gleiche Resultat ist auch in einem Ausblick auf die zukünftigen Ereignisse mit h\"o{}herer Anzahl von Pileup Kollisionen sichtbar. Darüber hinaus ist zu beobachten, dass eine gute Zuordnung von Teilchenspuren zu Vertizes erschwert wird.

\selectlanguage{english}
