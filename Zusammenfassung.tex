\selectlanguage{german}

\phantomsection\addcontentsline{toc}{chapter}{Kurzfassung}
\chapter*{Kurzfassung}
\markboth{Kurzfassung}{Kurzfassung}

Die hier vorgestellte Arbeit wurde im Kontext des CMS Experimentes am LHC angefertigt. Dieser Beschleuniger erreichte schon zum Ende der ersten Phase der Laufzeit im Dezember 2012 eine instantane Luminosit\"a{}t, welche nah an die Design-Luminosit\"a{}t von 1\ten{34}\percms{} heranreichte. Dies hatte unter anderem zur Folge, dass es bei einer Kollision von zwei Protonpaketen im Mittel zu 20 Proton-Proton Interaktionen kam. Die Verbesserungen, die zur Zeit am Beschleuniger durchgef\"u{}hrt werden, werden diese Anzahl der Kollisionen noch weiter vergr\"o{}\ss{}ern. Das bedeutet, dass es bei einer Kollision von zwei Protonpaketen neben einer m\"o{}glichen harten, physikalisch interessanten Kollision (Signal) viele weiche, uninteressante Interaktionen gibt. Hinzu kommt die erh\"o{}hte Schwerpunktsenergie von bis zu 14\TeV{}, welche die Anzahl der erzeugten Spuren pro Vertex vergr\"o{}\ss{}ert. Dieser Untergrund (auch Pileup genannt) kann dazu f\"u{}hren, dass die Aufl\"o{}sung der rekonstruierten physikalischen Objekte der interessanten Kollisionen verschlechtert wird. \\
Zur Zeit gibt es beim CMS Experiment zwei \"a{}hnliche Ans\"a{}tze, deren Ziel es ist, die Teilchenspuren, die nicht von der Signal-Interaktion kommen, wieder aus der Kollektion zu entfernen. Die hier vorgestellte Arbeit versucht, diese




Um den Einflu\ss{} von Teilchenspuren zu verringern, welche von Pileup Vertices kommen, wird mit Hilfe von Informationen des Spurdetektors eine association map gebildet, die jeder Teilchenspur einen rekonstruierten Vertex zuordnet. Zur Analyse der Signalteilchenspuren werden jene Teilchenspuren verwendet, die der h\"a{}rtesten Kollision zugeordnet wurden.


\selectlanguage{english}