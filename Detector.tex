\chapter{LHC and CMS \label{sec:LHCCMS}}

\section{The Large Hadron Collider \label{LHCCMSLHC}}

The Large Hadron Collider (LHC) is a proton proton collider located at the the CERN close to Geneva at the French-Swiss border. It is an underground ring accelator with a circumference of about 27\km{}. The tunnels in which the LHC is placed have already been build for the previous accelerator LEP, an eletrcon positron collider. A sketch can be seen in Figure~\ref{plot:LCHSketch}. Furthermore, the SPS as one of the pre accelerators can be seen. From this accelerator the protons are put into the LHC with an energy of 450\GeV{}. The foreseen energy of the LHC itself is 7\TeV{} per proton. During the first runtime of the LHC from 2010 until 2012 the LHC ran first with an energy of 3.5\TeV{} which was increased to 4\TeV{} for the final year. Additionally, at the end of every yeear the LHC also accelerated lead nuclei. Hence, also protons and lead nuclei or only lead nuclei can collide. \\
To bend the accelerated protons or lead nuclei superconductiong dipole magnets made of Nb-Ti are used. This requires cooling down to temperatures below 2\K{}. To focus the particle bunches in one direction magnets of higher order are used like quadrupole magnets. The acceleration itself is done with superconducting cavities made of copper with sputtered niobium. These are again cooled down with liquid helium.  This is reached by using superfluid helium. Through all these components the beam pipe is passing. At each interaction point the diameter of this beam pipe is reduced to around 6\cm{}. Around these interaction points the detectors are build up. Further information about the LHC can be found in~\citen{Bruning:782076}.

\begin{figure}[!ht]
  \centering
  \includegraphics[width=0.55\textwidth]{Detector/Sketch_LHC}
  \caption[Sketch of the LHC]{A sketch of the LHC and the four experiments as they are located at the French-Swiss border taken from~\cite{Team:40525}. \label{plot:LCHSketch}}
\end{figure}

Also shown in Figure~\ref{plot:LCHSketch} are the four main experiments located around the ring. Which are:
\begin{itemize}
\item{ALICE}: \textbf{A} \textbf{L}arge \textbf{I}on \textbf{C}ollider \textbf{E}xperiment, an experiment build for the study of the quark-gluon-plasma which is produced in lead-lead collisions. With this the conditions shortly after the big bang are reproduced. More information about the detector can be found in~\citen{ALICETDR}.
\item{ATLAS}: \textbf{A} \textbf{T}oroidol \textbf{L}HC \textbf{A}pparatu\textbf{S}, which one of the two multi purpose experiments. The cylindrical detector with a length of 46\m{} and a diameter of 25\m{} is the largest of the four LHC detectors. For more informatioin see~\citen{ATLASTDR1} and~\citen{ATLASTDR2}.
\item{CMS}: \textbf{C}ompact \textbf{M}uon \textbf{S}olenoid, which is the other multi purpose experiment. Compared to the ATLAS detector it is smaller but has a much higher weight. It will be more described later.
\item{LHCb}: \textbf{L}arge \textbf{H}adron \textbf{C}ollider \textbf{b}eauty, which is specialized for the study of the bottom (beauty) quark. For this, the innermost layer of the detector is very close to the interaction region of the protons. Furthermore, the detector has an asymmetrical design. More information can be found in~\citen{LHCb:630827}.
\end{itemize}


\subsection{Luminosity and Proton-Proton-Interactions \label{sec:IntroLumiPPI}}

One key parameter of the LHC is the delivered instantaneous luminosity. It depends only on the machine parameters and can be shown as follows
\begin{equation}
L = \frac{N_{b}^{2}n_{b}f_{rev}\gamma{}_{r}}{4\pi\epsilon{}_{n}\beta^{^\ast}}F
\label{eq:Luminosity}
\end{equation}
where
\begin{itemize}
\item{$N_{b}^{2}$} is the number of particles per bunch, design value is $1.15 \times 10^{11}$
\item{$n_{b}$} is the number of bunches per beam, design value is 2808, during the first run time it was only half of it
\item{$f_{rev}$} is the revolution frequency, which is in the order of $14-15\,kHz$
\item{$\gamma_{r}$} is the relativistic gamma factor
\item{$\epsilon{}_{n}$} is the normalzied transverse beam emittance, design value is $3.75\,\mum{}\rad$, during the first run time it was around $2.4\,\mum{}\rad$
\item{$\beta^{^\ast}$} is the amplitude function around the collision point, its in the order of 60\cm{}
\item{$F$} is the geometric luminosity reduction factor coming from the crossing angle of the two colliding proton bunches, design value is 0.836 at a crossing angle of 285\murad{}.
\end{itemize}

Putting all these values together the design instantaneous luminosity is around \LHigh{} for proton-proton runs. This can be translated into 10\hertzpernbarn{}. As shown in Figure~\ref{plot:IntroInstLumi} during the first run period an instantaneous luminosity greater than 7\hertzpernbarn{} has been reached.

\begin{figure}[!Hhtb]
  \centering
  \includegraphics[width=0.55\textwidth]{Intro/peak_lumi_per_day_pp_2012}
  \caption[Instantaneous luminosity at CMS]{The instantaneaous luminosity as delivered to CMS from the LHC accelerator during the proton-proton run period in 2012. \label{plot:IntroInstLumi}}
\end{figure}

To get  a more suitable parameter this instantaneous luminosity needs to be translated into proton-proton interactions per bunch crossing. For this, the total cross section for proton-proton interactions is taken from Figure~\ref{plot:IntroTotalCross}. At a center-of-mass energy of 8\TeV it is assumed to be approximately 90\mb. Together with the bunch spacing of 50\ns it leads for the number of interaction $n_{int}$ to
\begin{eqnarray}
n_{int} & \approx & 7\hertzpernbarn \cdot 90\mb \cdot 50\ns \nonumber \\
 & \approx & 31.5 \nonumber
\end{eqnarray}

Hence, the expected number of proton-proton interaction per bunch crossing during the first run period of the LHC is expected to be in the order of 30.

\begin{figure}[!Hhtb]
    \centering
    \includegraphics[width=0.7\textwidth]{Intro/ppXsection}
    \caption[Total and elastic cross section for pp interactions]{The total and elastic cross section for proton-proton collisions as a function of the center-of-mass energy. Taken from reference ~\citen{Beringer:1900zz}.\label{plot:IntroTotalCross}}
\end{figure}

\section{The Compact Muon Solenoid Experiment \label{LHCCMSCMS}}

\subsection{The CMS Detector \label{LHCCMSCMSDet}}

At interaction point five of the LHC the CMS detector is located. The cylindrical shaped detector with a length of 29\m{} and a diameter of 15\m{} is placed in a cavern around 100\m{} under ground. The total weight of the detector is around 14000 tonnes. An artistic sketch of the whole detector is shown in Figure~\ref{plot:LHCCMSWholeCMS}. As it can be seen the general installatioin of the sub-detectors is onion shaped. One section of the CMS detector with the layers of all sub-detectors is displayed in Figure~\ref{plot:LHCCMSOnionCMS}. The gap in the middle is filled by the solenoid magnet. Starting from the innermost point after the beampipe the sub-detectors are arranged in the following way: Pixel and strip tracker, electromagnetic calorimeter, hadronic calorimeter, magnet and muon chambers. These subdetectors are working at different temperatures which are reach by using liquid helium in cooling pipes.

\begin{figure}[!Hhtb]
  \centering
  \includegraphics[width=0.7\textwidth]{Detector/Sketch_CMS}
  \caption[Artistic sketch of the CMS detector]{An artistic sketch of the CMS detector. Taken from~\citen{CMSPlots}. \label{plot:LHCCMSWholeCMS}}
\end{figure}

\begin{figure}[!Hhtb]
  \centering
  \includegraphics[width=0.7\textwidth]{Detector/CMS_Onion}
  \caption[One section of the CMS detector with all layers]{One section of the CMS detector with all layers from the sub-detectors togtehr with the beam pipe. The magnet is placed in the gap. Taken from~\citen{CMSPlots}. \label{plot:LHCCMSOnionCMS}}
\end{figure}

\subsubsection{Silicon Tracker}

The innermost part of the CMS detector is the tracking system. It has a cylindrical shape with a length of 5.8\m{} and a diameter of 2.5\m{}. With this the acceptance of the tracker reaches a pseudorapidity of 2.5. The modules of the tracker are made of silicon. On the one hand, modules made of silicon can provide a very high granularity and a fast readout. This is needed since at its design parameters the LHC will collide two proton bunches every 25\ns{} with at all around 1000 created particles per event. Furthermore, silicon is expected to withstand these conditions with such a high radiation for a relatively high lifetime. On the other hand, using silicon modules leads to a high amount af readout electronics and cooling devices. This increases the probability of photon conversion or nuclear interactions. mor information about the choise which sensor technology has beeen chosen can be found elsewhere~\cite{Chatrchyan:1129810}. In total about 200\ms{} of active silicon area is build in the CMS tracker making it the largest silicon tracker ever build.\\
As demonstrated in the sketches shown in Figure~\ref{plot:LHCCMSTrackerCMS} the tracker can be further divided into a Pixel and a Strip detector. A schematic cross section through the CMS tracker can be seen in Figure~\ref{plot:LHCCMSTrackerCrossCMS}. Also there, the position of the Pixel and Strip tracker is ilustrated.

\begin{figure}[!Hhtb]
    \centering
    \includegraphics[width=0.45\textwidth]{Detector/CMS_Tracker}
    \includegraphics[width=0.45\textwidth]{Detector/CMS_PixelTracker}
    \caption[Sketches of the CMS Tracker]{A sketch of the whole CMS silicon Tracker (left hand plot) and a zoom to the Pixel Tracker and its enclosure only (right hand plot). Taken from~\citen{CMSPlots}. \label{plot:LHCCMSTrackerCMS}}
\end{figure}

\begin{figure}[!Hhtb]
    \centering
    \includegraphics[width=0.8\textwidth]{Detector/Sketch_CMS_Crosssection}
    \caption[Schematic cross section of the CMS Tracker]{A schematic cross section of the CMS tracker including the Pixel and Strip detector. Modules which are marked with double lines deliver stereo hits. Taken from~\citen{Chatrchyan:1129810}. \label{plot:LHCCMSTrackerCrossCMS}}
\end{figure}

\paragraph{Pixel Tracker}
To enable a good resolution of the reconstrcution of the track impact parameter the cells of the pixel tracker need to have a very small size. To achieve this a size has been chosen of $\text{100}\,\times\,\text{150}\,\mu\text{m}^{\text{2}}$ per pixel. As it can be seen from Figure~\ref{plot:LHCCMSTrackerCrossCMS} there are three barrel layers of the pixel detector with a length of 53\cm{} at radii of 4.4, 7.3 and 10.2 \cm{}. Furthermore, at the end planes of these cylinder two pixel disks are located at z values of $\pm{}34.5$ and $\pm{}46.5\,\text{cm}$. In doing so, in the barrel region the cells are aligned in the $\phi{}$-z-plane and in the disks in the x-y-plane. One more psecial feature of teh placement of the pixel disks is that they tilted by 20\de{} looking like a turbine. The reason for this is the charge-sharing which comes due to geometrical effects of the bending of hte particles in the magnetic field.  With this arrangement three hits per track are expected on average. In total, the pixel detector covers an area of about 1\ms{} and has 66 million pixels mounted on 1440 modules.


\paragraph{Strip Tracker}
As already shown in Figure~\ref{plot:LHCCMSTrackerCrossCMS} the strip tracker consists of several parts. In the central region to barrels, namely the Tracker Inner Barrel (TIB) and the Tracker Outer Barrel (TOB), are placed. On the end planed the Tracker Innder Disks (TID) and the Tracker End Caps (TEC) are installed. All these parts are equipped with several different silicon modules. For the two barrels rectangular sensors are chosen while there are 11 wedge-shaped sensors for the TID and TEC. One further difference is the thickness. While those modules which are close to the interaction region at the TIB, the TID and on the four inner rings of the TEC thin sensors with a thickness of 320\mum{} are used the other modules are 500\mum{} thick. These thicker modules can have a bow which is required to be less than 100\mum{}. These different thicknesses are due to the particle flux and the resulting signal to noise ratio. Finally, another important difference between the modules is the width. They can have either 512 or 768 strips with a pitch of 80$-$120\mum{}. In general it can be said the farer the sensors are ways from the interaction region the thicker, the mmore strip and the larger strip pitch they have. Due to the strip design the modules are able to measure the impact point of the track with a high accuracy in one dimension only. To provide also two dimensional information about the impact point at some places two modules are paired together with an angular difference of 100\mrad{}.\\
In the central region the TIB consists of four cylinders with a length of 140\cm{} placed at radii of 255.0, 339.0, 418.5 and 498.0\mm{}. This tracker part is followed by the TOB which is the only tracker sub-detector which is fully equipped with thick sensors. It consists of six cylindrical layers with a length of 236\cm{}. The layers are placed at radii of 608, 692, 780, 868, 956 and 1080\mm{}. The modules are arranged in that was that that the strips are parallel to the z direction. Hence the $\phi{}$ position of the hit can be measured with highest accuracy. For both, TIB and TOB, the two innermost layers are equipped with double-sided modules giving also information of the z position of the hits.\\
Next to the TIB the TID is placed. It consists of three disks at z positions of $\pm{}$80 to $\pm$90\cm{} . These disks are identical and assembled with three rings of modules spanning a radius from 200 to 500\mm{}. Due to this orientation again the $\phi{}$ position of the hit has the highest accuracy. The two innermost rings of each wheel are equipped with double-sided modules. In doing so, the radius of the position of the double-sided modules of TIB and TID are similar. Addotionally, the radius of the hit from the beam axis can be measured at thw double-sided modules. Last, the TEC consists of 9 endcaps at z positions from $\pm{}$124 to $\pm{}$ 280\cm{}. The cover a radius from 220 to 1135\mm{}. Depending on the z position the endcaps are equipped with four to seven rings of modules. The double-sided modules are again placed in that way the the distance to the beam axis is the same as for those in the TIB and TOB. \\
In total, 15 148 strip detectotr modules are installed leading to 24 244 channels from 9.3 million strips. In Figure~\ref{plot:LHCCMSTrackerRadiationLength} the material budget in terms of radiation length is shown for the whole tracker and divided into the sub-detectors. As it can be seen for the central region it is between 0.4 and 0.6 but for a pseudorapidity between 1. and 2. is rises up to 1.8. For electrons the radiation length is the typical length in which an high energetic electron loses its energy due to Bremsstrahlung down to 1/e. For photons it is defined as 7/9 of the mean free path for pair production of a high energetic photon. This has as a consequence that about $75\,\%$ of all photons convert within the tracker volume.\\
The resulting resolution of the position of a hit is in mostly between 10 and 50\mum{} \cite{CMS-PAPER-TRK-11-001}.

\begin{figure}[!Hhtb]
    \centering
    \includegraphics[width=0.55\textwidth]{Detector/Tracker_RadiationLength}
    \caption[Material budget of the CMS Tracker]{The material budget of the CMS tracker in multiples of the radiation length as a function of the pseudorapidity. It is further broken down to the sub-detectors. Taken from~\citen{Chatrchyan:1129810}. \label{plot:LHCCMSTrackerRadiationLength}}
\end{figure}

\subsubsection{Electromagnetic Calorimeter}

To measure the elecromagnetic energy right behind the tracker at a hermetic and homogeneous calorimeter made of more than 75000 lead tungstate crystals is used. It is again cylindric with a length of about 630\cm{} and a diameter of 354\cm. The barral of it can be seen on the left hand plot in Figure~\ref{plot:LHCCMSECalCMS} while a typical endcap of a calorimeter is shown in the right hand plot of Figure~\ref{plot:LHCCMSHCalCMS}. The transition region from the barrel to the endcaps is at a pseudorapidity of 1.479. The individual crystals have been arranged in that way that all face the interaction point as shown on the right hand plot of Figure~\ref{plot:LHCCMSECalCMS}. Lead tungstate has been chosen because of its high density (8.28\gcmc{}), short radiation length in the order of 0.89\cm ans a small Moli\`ere radius of 2.2\cm{}. This Moli\`ere radius is a characteristic constant of a material. A cylinder with this radiu contains on average $90\,\%$ of the energy deposition of a electromagnetic shower. The crystals have a pyramidal shape with a cross-section of 22 $\times{}$ 22\mms{} at the inner face and 26 $\times{}$ 26\mms{} at the rear face. The length of one crystals is 23\cm corresponding to 25.8 radiation lengths.

\begin{figure}[!Hhtb]
  \centering
  \includegraphics[width=0.45\textwidth]{Detector/CMS_ECal}
  \includegraphics[width=0.45\textwidth]{Detector/CMS_ECal_Modules}
  \caption[Sketches of the electromagnetic calorimeter of CMS]{A sketch showing the barrel of the CMS electromagnetic calorimeter (left hand pot) and one slice of the modules (right hand side). Taken from~\citen{CMSPlots}. \label{plot:LHCCMSECalCMS}}
\end{figure}

\subsubsection{Hadronic Calorimeter}

The hadronic calorimeter is the last subdetector which is placed inside of the magnet for the most parts. It is not only important for the measurement of hadronic jets but also for the calculation of the missing energy. The general structure is illustrated in Figure~\ref{plot:LHCCMSHCalCross}. There are four subparts of the hadronic calorimeter.\\
The two subdetectors inside the magnets, the hadron barrel (HB) and hadron endcaps (HE), are ssmapling calorimeters. This is due to the fixed amount of volume between the elecromagnetic calorimeer and the magnet. Furthermore, the produced magnetic field restrict also the material of which the hadronic calorimeter is build. Further requierements on the modules are a fast read out. Therefore, brass has been chosen as the absorber material and plastic scintillator read out by hybrid photodiods at the active medium. The barrel covers a radius region from 1.77 to 2.95\m{} up to a pseudorapidity of $\left|\eta{}\right| \leq{} 1.3$. The endcaps enlarge the pseudorapidity range to $\left|\eta{}\right| \leq{} 3$. While for the HB the granularity is always $\Delta{}\eta\,\times\,\Delta{}\phi{} = 0.087 \times{} 0.087$ in the HE the granularity changes to $\Delta{}\eta\,\times\,\Delta{}\phi{} \approx 0.17 \times{} 0.17$ for $\left|\eta{}\right| \geq{} 1.6$. These towers consists of 16 layers of scintillators and absorber starting and ending with a scintillator. For the HB it is as follows. While the first and the last layer of the scintillator is 9\mm{} thick all others are 3.7\mm{} thick. For the absorber its a bit more complex. The first and last layer are made of steel with a thickness of 40\mm{} and 75\mm{}, respectively. The first eight brass layers have a thickness of 50.5\mm{} while the last eight are 56.5\mm{} thick. For the HE all brass plates are 79\mm{} thick and except for the first scintillator layer all other layers are 3.7\mm{} thick. This first scintillator layeer has a thickness of 9\mm{}. A sketch of the HB and HE can be seen in Figure~\ref{plot:LHCCMSHCalCMS}. In total the thickness of the inner calorimeters in terms of interaction length is about five for $\eta{}=0$ increasing to about ten for $\left|\eta{}\right|\approx{}3$.\\
Hence, in the central region the HB does not provide enough stopping power for high energetic jets. Therefore, behind the magnet the hadronic calorimeter is extended with a tail catcher named Hadron Outer (HO). Except for the central region ($\left|z\right| \leq{} 1.343\,m$) it consist of one (two for the central region) layer of plastic scintillator read out by hybrid photodiods. The granularity is the same as for the HB. The available space between the magnet and the first layer of the muon chamber is in the order of 4\cm{} which leads to about 1.6\cm{} for the detector layer. Unfortunately, this detector has not been used frequently at the event reconstrcution.
The fourth part of the hadronic calorimeter is the hadron forward (HF) detector. It is the detector component with covering the  highest $\eta{}$ range up to 5. It is a Cherenkov detector made of quartz fibres. These fibras are build up of a 600\mum{} thick fused-silica core which is sorrounded by a 30\mum{} thick polmer hard-cladding. For protection a 170\mum{} thick acrylate buffer is mounted. The HF itself is a cylinder with a radius of 130\cm{} at a distance of 11.2\m{} from the interaction point. The fibres are running parallel to the beam axis. The information of this detector is also used to calculate the luminosity which is a very important part.

\begin{figure}[!Hhtb]
    \centering
    \includegraphics[width=0.7\textwidth]{Detector/CMS_HCal_Cross}
    \caption[Cross section of the CMS hadronic calorimeter]{The cross section of one quearter of the CMS hadronic calorimeter showing the hadron barrel (HB), the hadron endcaps (HE), the hadron outer (HO) and the hadron forward (HF) detector. Dashed lines symbolize fixed values of the pseudorapidity. Taken from~\citen{Chatrchyan:1129810}. \label{plot:LHCCMSHCalCross}}
\end{figure}

\begin{figure}[!Hhtb]
  \centering
  \includegraphics[width=0.45\textwidth]{Detector/CMS_HCal}
  \includegraphics[width=0.45\textwidth]{Detector/CMS_ECal_EndCap}
  \caption[Sketches of the hadronic calorimeter of CMS]{A sketch of the hadronic smaple calorimeter of the CMS detector (left hand plot) and one of a typical endcap of the electronmagnetic of hadronic calorimeter (right hand plot). Taken from~\citen{CMSPlots}. \label{plot:LHCCMSHCalCMS}}
\end{figure}

\subsubsection{Superconducting Magnet}

In Figure~\ref{plot:LHCCMSMagnetCMS} a sketch is shown of th e superconductiong magnet. It has length of 12.5\m{} and a diameter of 6\m{}. It is designed to reach a uniform axial 4 Tesla field in the inside with an inductance of 14.2\unit{H} and a stored energy of 2.6\unit{GJ}. Together with its weight of 220 tonnes it reaches the best energy to mass ratio of all magnets used in particle physics. On the outside the field it returned through a yoke. This yoke alone has a weight of 10000 tonnes. The magnet itself if made of NbTi running at a Temnperature of 4.6\K{}.

\begin{figure}[!Hhtb]
  \centering
  \includegraphics[width=0.7\textwidth]{Detector/CMS_Magnet}
  \caption[Sketch of the solenoid magnet]{A sketch showing the solenoid magnet of the CMS detector and the beam pipe with a human for size comparison. Taken from~\citen{CMSPlots}. \label{plot:LHCCMSMagnetCMS}}
\end{figure}

\subsubsection{Muon Chambers}

The outermost subdetectors are the muon chambers. As it is already implied by the middle name of CMS detecting muons is one of the key purposes. In total an active area of about 25000\ms{} is used in the muon chambers. There three different typed of detectors used as it also can be seen in Figure~\ref{plot:LHCCMSMuonChambersCMS}. In the barrel region up to a pseudorapidity of $\left|\eta{}\right| \leq{} 1.2$ drift tubes (DT) are used. In 250 drift chambers about 172000 sensitive wires are plugged. These wires have a length of 2.4\m{} measuring the hit position in the $r-\phi{}$-plane. The DT are 2.1\cm{} thick and use a gas mixture of $85\,\%$ Ar and $15\,\%$ $\text{CO}_{\text{2}}$. The DT are mainly used for the tracking of the muons. In the forward region cathode strip chambers (CSC) are used for this purpose. That are multiwire proportional chambers with six anode wire planes interlaeved among seven cathode panels. The size of such a CSC can go up to $3.4\,\times{}\,1.5\ms{}$. For triggering resistive plate chambers (RPC) are used. That are gaseous parallel-plate detectors. Its biggest advantage is the very good time resolution which is needed for triggering. In the barrrel regioin six layers of these RPC's are embedded in the yoke of magnet. During the first run period of the LHC in the endcap region three RPC layers are installed covering a region up to $\left|\eta{}\right| = 1.6$.

\begin{figure}[!Hhtb]
  \centering
  \includegraphics[width=0.7\textwidth]{Detector/CMS_MuonChambers}
  \caption[Sketch of the muon chambers of CMS]{A sketch of the muon chambers of the CMS detector. The drift tubes (DT) are shown in red followed by the resistive plate chambers (RPC). Perpendicular to the beam pipe the cathode strip chambers (CSC) and again RPC's are mounted. Taken from~\citen{CMSPlots}. \label{plot:LHCCMSMuonChambersCMS}}
\end{figure}


\subsection{The Event Reconstruction \label{LHCCMSEventReco}}

\subsubsection{Trigger}

At its design luminosity the LHC collides proton bunches at a frequency of 40\unit{MHz}. During the first run period it has been at 20\unit{MHz}. Considering now the millions of channels which are read out during the reconstruction of one event it can be seen that it is impossible to reconstruct each event. Therefore, a drastic event reduction neeeds to be done. This tasks is performed by a two stage trigger. While the first one so-called \Lone{} (L1) trigger takes only the information from the calorimeters and the muon chambers at the following High-Level trigger (HLT) also a very rough reconstruction of the whole event is done. The design aim of the L1 trigger is to reduce the incomming 20\unit{MHz} to a outgoing rate of $30-100\unit{kHz}$. The data rate after the L1 trigger is in th eorder of 100\GBytes{}. More dedicated information about the two levels of the trigger and the following data acquisition can be found in~\citen{Chatrchyan:1129810} adn~\citen{Cittolin:578006}.

\subsubsection{Particle and Track Reconstruction}

The key towards of complete and accurate reconstruction of the whole event at the CMS exxperiment is the particle-flow algorithmus. In it all stable particles are tried to be reconstructed and identified using a thorough combinationi of all subdetectors. The created list of particles is then used to build up jets or calculate the missing transverse energy. A very detailed description of the algorithm can be found in~\citen{CMS-PAS-PFT-09-001}. This thesis will only focus on some aspects which are very important for the following work which are the reconstruction of charged tracks in the tracker and the reconstruction of primary and secondary vertices.

\paragraph{Iterative Tracking}

At CMS the reconstruction of tracks from charged particles happens in several iterations. Starting with those tracks which are easiest to find for the following iterations all used hits are removed from the initial hit collection. In doing so, the combinatorical complexity is reduced allowing to reconstruct more difficult tracks. In each iteration three steps are executed. First, a collection of possible seed is created. Depending on the iteration only some of the subdetectors of the tracker are used. Based on these seeds then a Kalman Filter~\cite{Fruhwirth:1987fm} is run building up a trajectory of the possible track. Along this trajectory it is looked for more hits fitting to it. Finally, based on all these hits the track is fitted using again the Kalman filter and smoother. More information about the seeding and fitting of the tracks can be found in~\citen{CMS-PAPER-TRK-11-001}. Having executed this reconstruction tracks with a transverse momentum down to 0.1\GeVc{} can be reconstructed based on at least 5 hits. At the reconstruction electrons need special treatment since they undergo Bremsstrahlung and lose a large amount of their energy. This loss is highly non-Gaussian. Therefore, the Kalman filter does not work very well for electrons since it is optimal only when all vriables have Gaussian uncertainties. In order to account for this two methods have been developed using the information from the ECal as additional seeds (look at~\citen{CMS-PAS-PFT-10-003} and~\citen{CMS-PAS-EGM-10-004} for more information). In order now to include the non-Gaussian uncertainty the Kalman filter has been modified to the Gaussina Sum Filter (GSF) as it is described in~\citen{Adam:815410}. Havind executed the tracking a total efficiency above $90\,\%$ can be reached at a fake rate in the order of $10\,\%$. \\
One important outcome of the track reconstruction is the resolution of the vertex of the track, especially the resolution of the z position. As it will be explained in Section~\ref{sec:IntroSepResPV} the region where the tracks come from is much larger along the z axis than in the other two dimensions. Hence, a good precision of the z position of the origin of the track is necessary. The resolution is estimated based on simulated events and is calculated as follows. For every reconstructed track the best matching simulated track is taken. This best match is searched by comparing the number of shared hits. Also, a minimum number of shared hits is needed. From both tracks the z value of the position of closest approach to the interaction region is taken. The distance between these two values is then stored in a histogram. After this has been done for all analyzed tracks a gaussian is fitted to this histogram. Finally, to obtain the resolution of the z position of the tracks vertex the standard deviation of this gaussian is divided by $\sqrt{2}$.  As shown in Figure~\ref{plot:IntroTrackRes} the averaged resolution of the z position of the reconstructed track lies between 30\mum and can go up to some millimeter for high $\left|\eta\right|$ and low \pt{}.

\begin{figure}[!Hhtb]
    \centering
    \includegraphics[width=0.55\textwidth]{Intro/TrackRes_dz_Vs_Pt}
    \caption[Track resolution \vs pseudo rapidity for different bins of transverse momentum]{The distribution of the track resolution in the transverse impact parameter $d_{z}$ as a function of the pseudorapidity for different bins of transverse momentum. The tracks have been divided into three regions of pseudorapidity, in the barrel region $\left[0., 0.9\right]$, the transition region $\left(0.9, 1.4\right]$ and the endcap region $\left(1.4, 2.5\right]$\label{plot:IntroTrackRes}}
\end{figure}

\paragraph{Primary Vertex Reconstruction}

Another important collection are the reconstructed primary vertices. These are produced by clustering a subset of the reconstructed tracks into sets of tracks which are coming from nearly the same position. These tracks have to fulfill certain quality selections and have its origin close to the interaction region. The most important feature for the work presented here is the so-called track weight. This weight is calculated after the fit is finished and stands for the compatibility between track and the fitted vertex. It is greater than 0 for those tracks only which have been used at the fit of the respective vertex. The closer the value gets to 1 the more compatible the track is with the vertex. As said above only a subset of the reconstructed track is used for the production of the primary vertices. As it can be seen in Figure~\ref{plot:IntroTrackWeight} only about $60\%$ of all reconstructed tracks have a track weight at any vertex.  More information about the technique of the vertex fit can be found elsewhere~\cite{CMS-PAPER-TRK-11-001}.

\begin{figure}[!Hhtb]
    \centering
    \includegraphics[width=0.45\textwidth]{Intro/VertexNum_Vs_Ntrks}
    \includegraphics[width=0.45\textwidth]{Intro/TrackWeightCheck}
    \caption[Distribution of number of tracks used for fitting the vertices and fraction of reconstructed tracks which have a track weight at any primary vertex]{The distribution of the number of tracks used for fitting a particular vertex (left hand side) and the fraction of reconstructed tracks which have a track weight greater than 0 at the chosen signal vertex, at any primary vertex or at no vertex at all as a function of the pseudorapidity (right hand side). \label{plot:IntroTrackWeight}}
\end{figure}

As shown in Section~\ref{sec:IntroLumiPPI} in one bunch crossing on average 30 proton-proton interactions are expected if th e LHC runs at its maximum for the first run period. The space in which all these collisions happen is the so-called beam-spot. The shape of the beam-spot can be described as a cylinder lying centrally along the z-axis. This cylinder has a diameter of about 120\mum and a total length of about 30\cm. Therefore, the biggest separation of the primary vertices is expected to be along the z axis compared to the other axis. The probability for a collision to be at a certain position z is shown in Figure~\ref{plot:IntroVertexPos}. The shape of the probability can be well-described as gaussian with a width of about 5\cm.

\begin{figure}[!Hhtb]
    \centering
    \includegraphics[width=0.55\textwidth]{Intro/VertexNum_Vs_Z}
    \caption[Vertex distribution along z]{The distribution for the reconstructed primary vertices along the z axis for the last run period in 2012 (normalized to an integral of 1). \label{plot:IntroVertexPos}}
\end{figure}

The mean separation of the primary vertices in one bunch crossing is therefore strongly dependent on the number of collisions. For the data in Figure~\ref{plot:IntroVertexSep}, the mean separation is shown against the number of reconstructed vertices of the bunch crossing. The separation has been defined as the distance of one primary vertex to the closest other one. For bunch crossings with more than 30 reconstructed vertices the mean separation becomes less than 5\mm. The same holds for an instantaneous luminosity of greater than 6\hertzpernbarn. A more detailed look of the primary vertex separation can be found in Figure~\ref{plot:IntroVertexSep2D}. As it can be seen the separation of the primary vertices can go down below 1\mm in the central region.

\begin{figure}[!Hhtb]
    \centering
    \includegraphics[width=0.45\textwidth]{Intro/VertexSep_Vs_Npu}
    \includegraphics[width=0.45\textwidth]{Intro/VertexSep_Vs_InstLumi}
    \caption[Mean vertex separation \vs number of reconstructed vertices and \vs instantaneous luminosity]{The distribution for the mean separation of the primary vertices \vs the number of reconstructed vertices (left hand side) and \vs the instantaneous luminosity for one run of the last run period in 2012. \label{plot:IntroVertexSep}}
\end{figure}

\begin{figure}[!Hhtb]
    \centering
    \includegraphics[width=0.45\textwidth]{Intro/VertexSep_Vs_Npu_Vs_Z}
    \includegraphics[width=0.45\textwidth]{Intro/VertexNum_Vs_Z_Vs_Sep}
    \caption[Vertex separation \vs number of reconstructed vertices and \vs z position of the vertex. Number of signal vertices \vs z position and separation for events with 20 underlying pileup events]{The distribution for the averaged separation of the signal vertex \vs the number of reconstructed vertices and the z position (left hand side) and the number of signal vertices \vs z position and separation for events with 20 underlying pileup events for one run of the last run period in 2012. \label{plot:IntroVertexSep2D}}
\end{figure}

To estimate the resolution of the primary vertex reconstruction the "split method" has been applied on data from one run at the end of the first run period in 2012. A more detailed description of this method can be found in reference~\citen{CMS-PAPER-TRK-11-001}. First, all tracks from a certain vertex are split equally into two different sets. In this process, one tries to ensure that both track sets have, on average, the same kinematic properties. Based on these two sets two new vertices are fitted independently using the adaptive vertex fitter. The distance of the position of these two vertices is then stored. After applying this procedure to all primary vertices a gaussian is fitted to the distribution of that distance. To obtain the resolution of the primary vertices the standard deviation of the gaussian is divided by $\sqrt{2}$. In Figure~\ref{plot:IntroVertexRes} the primary vertex resolution in z is shown. As it can be seen the resolution depends strongly on the number of tracks that have been used to fit the primary vertex. The typical range for this resolution lies between 200\mum for about 10 tracks and around 40\mum for more than 60 tracks.

\begin{figure}[!Hhtb]
    \centering
    \includegraphics[width=0.55\textwidth]{Intro/VertexResolution_Vs_nTrks}
    \caption[Vertex resolution \vs number of used tracks]{The distribution for the resolution in z of the primary vertices \vs the number of tracks that were used to fit the primary vertex for one run of the last run period in 2012. \label{plot:IntroVertexRes}}
\end{figure}

Thus, the complete picture for the distribution of the primary vertices is that the typical separation lies in the millimeter region while the resolution is two orders of magnitude smaller. On the other hand the resolution of the track reconstruction lies somewhere in between.\\

To pick the physically most interesting interactions the vertex with the highest $\pt^{2}$ sum is typically chosen. In the calculation for a vertex, the $\pt^{2}$ from all track which were used fitting the vertex is summed up. In order to account for misreconstructed track momenta the \pt-error is first subtracted from the \pt and only if the reduced \pt is still greater than 0 it is squared and added to the sum. Figure~\ref{plot:IntroSigVertexProb} shows the probability that the chosen primary vertex is the closest reconstructed vertex to the simulated signal interaction. For events with more than 40 underlying pileup events the efficiency can fall below $90\,{}\%{}$.

\begin{figure}[!Hhtb]
    \centering
    \includegraphics[width=0.55\textwidth]{Intro/FirstVertexCheck_Vs_Npu}
    \caption[Probability to define the correct vertex as signal]{The percentage for events in which the correct primary vertex is defined as the signal vertex as a function of number of reconstructed vertices. \label{plot:IntroSigVertexProb}}
\end{figure}

\paragraph{Secondary Vertex Reconstruction}

One more ingredient to the presented work are the reconstructed secondary vertices. Part of this work are namely the photon conversions, the $\text{V}^{\text{0}}$ decays standing for \PKzS{} and \PgL{}, secondary vertices coming from nuclear interactions and another fitter tuned to the fit of vertices coming from the decay of B hadrons. \\
A detailed description of the reconstruction of photon conversions can be found elsewhere~\cite{GiordanoConversion} or~\cite{CMS-PAS-EGM-10-005}. The basic principle is to find pairs of oppositely charged particles which have vertices close to each other and are parallel at that point. Additionally, invariant mass of the pair has to fit to a photon. Based on these two tracks a vertex is then fitted. \\
The reconstrcution of $\text{V}^{\text{0}}$ decays is somehow similar. It is described in~\citen{v0paper}. It aims to find secondary vertices as coming from $\PKzS{} \rightarrow{} \Pgpm{}\Pgpp{}$ and $\PgL{} \rightarrow{} p\Pgpm{}$. Thus, it requires the presence of two oppositely charged particles having their verices at a similar position. Furthermore, the invariant mass of the pair has to be within a certain window around the nominal \PKzS{} (\PgL{}) mass. \\
The reconstruction of nuclear interactions is described in~\citen{CMS-PAS-TRK-10-001}. It is a much more genral approach trying to link all pairs of tracks whose distance at the closest approach is small enough.  Based on the these pairs several secondary vertices are fitted. If the invariant mass of the pair matches to a photon or a $\text{V}^{\text{0}}$ the secondary vertex is removed from the list. \\
The last producer of secondary vertices, namely the inclusive vertex finder, has been established to reconstruct the vertex of decays of B hadrons which mostly comes along with jets. A description can be found in~\citen{ivfPaper}.


\subsection{Reconstructed Proton-Proton Interactions}

The distribution of the number of reconstructed proton-proton interactions per bunch crossing is shown in Figure~\ref{plot:IntroPileupDistr}. From this it can be seen that at CMS the mean of the number of proton-proton interaction for the first run period was 21. That means that on average in an event with one hard interaction about 20 soft interaction happen. The tail reaches up to 40 additional interactions. Comparing this to the expected number of 30 as calculated in Section~\ref{sec:IntroLumiPPI} a good agreement can be seen. The observed value is of course lower since the accelator did not always run on its maximum. This additional background is often referred to "pileup". It can have a severe influence on the performance of the reconstruction of the hard collisions (so-called signal process).

\begin{figure}[!Hhtb]
    \centering
    \includegraphics[width=0.55\textwidth]{Intro/pileup_pp_2012}
    \caption[Number of pp collisions]{The distribution for the number of proton proton collisions calculated based on Figure~\ref{plot:IntroInstLumi}. Taken from reference~\citen{CMS-Lumi-Public-Webpage}. \label{plot:IntroPileupDistr}}
\end{figure}

To illustrate the influence pileup can have Figure~\ref{plot:IntroOccuTracker} is shown. Both pictures display the occupancy of the CMS tracker for a simulated \ttbar event with 20 additional pileup events. The difference is that on the left hand side picture only the tracks coming from the signal vertex are displayed while on the picture on the right hand side all tracks displayed. Here, only tracks from charged particles are displayed since neutral particles do not leave hits in the tracker material. It is imaginable that these additional tracks affect the reconstruction of objects like jets or the missing transverse energy of the signal process.

\begin{figure}[!Hhtb]
    \centering
    \includegraphics[width=0.45\textwidth]{Intro/tracker_withoutPU}
    \includegraphics[width=0.45\textwidth]{Intro/tracker_withPU}
    \caption[Occupancy of the tracker with and without pileup]{Both pictures show the occupancy of the CMS tracker at a \ttbar event. On the left hand side without pileup and on the right hand side with 20 underlying pileup interactions. Here, all tracks are shown which leave at least three hits in the tracker material, have a \pt greater than 1\GeVc and a pseudorapidity of $\left|\eta\right|\leq2.4$. \label{plot:IntroOccuTracker}}
\end{figure}

\subsubsection{Current Pileup Subtraction Techniques \label{sec:IntroCurPST}}

First, it should be mentioned that the association of tracks to primary vertices can only be done for charged particles which leave hits in the tracker. For neutral particles it is more or less impossible to obtain a sufficient resolution on the momentum due to the much poorer spacial resolution of the calorimeter. In order to subtract tracks coming from pileup vertices, two different approaches have been developed. Both techniques take advantage of the track weight which is stored in each vertex as explained in Section~\ref{sec:IntroProdPV}. Furthermore, the techniques only divide the reconstructed vertices into one signal vertex and one group of pileup vertex. For the latter ones no individual association is done. Therefore, both approaches strongly rely on the correct identification of the signal vertex (see Section~\ref{sec:IntroSigVer}). Additionally, the clustering of the tracks to vertices can be faulty. \\
Another analogy is, that both approaches are executed after the so-called \textit{particle flow} identification has been applied. This makes it possible, that only for charged hadrons  the check for the track weight is done.  Only if those have a track weight at one of the pileup vertices they get subtracted. The difference in these two approaches shows up in the treatment of the charged hadrons which do not have a track weight at any primary vertex. All other particles are treated as signal, first. In following filters for muons and electrons a cut is applied on the distance from the track to the signal vertex. In doing so, pileup subtraction is executed for charged hadrons, muons and electrons. Hence, the new approach will be compared to the outcome of these associations and filters.

\paragraph{Jet/MET Approach \label{sec:IntroJM}}

In this approach, all charged hadrons which have no track weight at any primary vertex are treated as coming from the signal one. In doing so, the vertex with the highest \pt-sum is overvalued because about $40\%$ of all reconstructed charged hadrons are treated as signal always.

\paragraph{Muon/Egamma Approach \label{sec:IntroME}}

In this approach, charged hadrons which have no track weight at any primary vertex are assigned to the closest vertex in z. This approach takes advantage of the fact that the largest separation or the primary vertices is along the z axis.

\paragraph{Basic Ideas for a New Approach}

In order to reach a better performance in assigning tracks to primary vertices, a more individual search for the best matching vertex needs to be performed. As mentioned above no association to a particular pileup vertex takes place until now. Hence, afterwards it is impossible to obtain the tracks coming from one special pileup vertex. Especially tracks without a track weight at any vertex are not assigned. To provide the possibility to obtain the track assigned to any primary vertex an \textit{Association Map} is created. In this map for any primary vertex it is possible to get a set of tracks together with a number indicating the quality of this association. In an additional step, it is possible to obtain only those tracks assigned to the signal vertex. Thus, tracks considered as coming from pileup vertices are subtracted from the initial track collection.

