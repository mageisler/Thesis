%%
%% This is just the table of the cuts for the reco track selection
%%
%% Parameter | Value
%%

\begin{table}[h]
\begin{center}
\caption[Selection criteria applied on the reconstructed tracks]{The different parameters of the reconstructed tracks and the applied filters. The $\chi^{2}$ is a result of the fit of the track. The transverse impact parameter is the minimum distance of the track to the beam axis in the $x$-$y$ plane while the longitudinal one is the minimum distance of the track to the position of the nominal interaction region in z direction. Both values are chosen in a way that the whole tracker volume is covered. The interaction region itself is calculated based on all collected events during a period of 23 seconds. The explanation for the given track qualities can be found in reference~\citen{CMS-PAPER-TRK-11-001}.}
\label{tab:TARecoTrackFilter}

\begin{tabular}{c c }
Parameter & Value \\
\midrule
\pt & $\geq1.0\GeV$ \\
$\left| \eta \right|$ & $\leq 2.4$ \\
number of hits & $\geq 3$ \\
$\chi^{2}$ & $\leq 10\,000$ \\
transverse impact parameter & $ \leq 120\cm $ \\
longitudinal impact parameter & $\leq 280\cm $ \\
quality & \textit{high purity}, \textit{tight} or \textit{loose} \\

\end{tabular}

\end{center}
\end{table}