\phantomsection\addcontentsline{toc}{chapter}{Abstract}
\chapter*{Abstract}
\markboth{Abstract}{Abstract}

The presented thesis was made in the context of the CMS experiment at the LHC. This particle accelerator already reached at the end of the first run period in December 2012 an instantaneous luminosity, which came close to the design luminosity of 1\ten{34}\percms{}. Amongst others this had the result that on average 20 proton-proton interactions occured at the collision of two proton bunches. This means that in addition to a possible hard, physically interesting interaction (signal) many soft, uninteresting interactions took place. The improvements, which are carried out now (2014) at the accelerator, increase this number of collisions even further. Moreover, the increased center-of-mass energy of up to 14\TeV{} enlarges the number of tracks produced per vertex. This background (also called pileup) can lead to a deteriorated resolution of the reconstructed physical objects of the interesting interaction.\\
Currently, there are two similar approaches existing at the CMS experiment whose goal is to remove those tracks from the collection that do not originate from the signal interaction. The prepared thesis presents a new approach that attempts to achieve a better association based on more information of the tracker. Moreover, it is possible to assign a particle track to more than one vertex. For the following analyses of the signal tracks those are used that are associated with the hardest interactions. Through the different options of the new approach, in the assignment of signal tracks both higher efficiency and a better combination of efficiency and purity can be achieved compared to the two approaches. \\
To analyze the influence on following reconstructions the results of the relative isolation, the jet reconstruction, the calibration of the missing transverse energy and the tagging of so-called b jets were studied. While in the calibration of the missing transverse energy the new approach shows the best results, a simple statement about the reconstruction of the jets is not possible. Before correcting the momentum of the jets, the new approach shows a smaller dependence on most parameters. But compared to an existing approach also a smaller reconstruction of the momentum is seen. By multiple associations per track this difference can be reduced. It is similar with the identification of b jets. With more than one association, the new approach is at least as efficient as the previous approaches. \\
Furthermore, the results of the simulation were compared with actual data of the CMS experiment, if possible. For most observables a good agreement is visible. Likewise, here, the new approach gives the best performance in the calibration of the missing transverse energy. This result is also present in the outlook on the expected number of pileup interactions. Moreover, it shows that a good assignment of tracks to vertices becomes more difficult.
