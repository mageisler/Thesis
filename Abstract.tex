\phantomsection\addcontentsline{toc}{chapter}{Abstract}
\chapter*{Abstract}
\markboth{Abstract}{Abstract}

The present thesis focuses on the tracking performance of the CMS experiment at the LHC. Already at the end of the first run period in December 2012 the accelerator nearly reached an  instantaneous luminosity of 1\ten{34}\percms{}, resulting in about 20 proton-proton interactions on average per bunch crossing. This implies that in addition to a possible physically interesting interaction many soft, uninteresting interactions take place. The improvements at the accelerator, which are carried out now (2014), are expected to increase this number of collisions even further. Moreover, the increased center-of-mass energy of up to 14\TeV{} enlarges the number of tracks produced per vertex. This background can lead to a deteriorated resolution of the reconstructed physical objects of the hard interaction.\\
Currently, there are two similar approaches applied at the CMS experiment to remove those tracks from a collection of reconstructed particles that do not originate from the signal interaction. This work presents a new approach to achieve a better association based on more information of the tracker. Moreover, the possibility of assigning a particle track to more than one vertex is implemented. This analysis focuses on signal tracks, which are associated with the hardest interactions. Through different options of the new approach the performance of the assignment of signal tracks can be regulated. This yields in a better performance compared to the two established approaches. \\
The impact on final physics object performance is tested, for instance for the calibration of the missing transverse energy. Almost entirely, an improvement of the physical results is observed with the presented approach. \\
Furthermore, the results of the simulation are compared with actual data of the CMS experiment. For most observables a good agreement is visible. Also here, the new approach shows a significant improvement in the calibration of the missing transverse energy. Moreover, especially in scenarios with a higher number of interactions the new approach seems to provide significant improvement concerning that calibration. Nevertheless, a good assignment of tracks to vertices becomes more difficult after future upgrades.
