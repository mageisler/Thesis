\chapter{Summary and Outlook \label{sec:SandO}}

In this work a new approach of the subtraction of tracks coming from pileup vertices has been presented. As introduced in Chapter~\ref{sec:LHCCMS} until end of 2012 on average about 20 additional proton-proton interactions took place simultaneously to the one signal interaction. These underlying events can have severe impact on the reconstruction of all objects. To reduce this impact, two techniques have been developed in the past. Both techniques are very similar in that way that they only apply a selection based on the trajectory for muons and electrons. For charged hadrons that are not used for the fit of the primary vertices the techniques follow different approaches. In order to obtain better results a new method has been introduced in the scope of this work. It takes advantage of several reconstructed objects like secondary vertices and uses advanced techniques to find the best matching primary vertex for a particular track. In Chapter~\ref{sec:AssMap} the work flow of this association has been explained. It can be divided in three steps. Starting with a test whether the track is used for the fit of the primary vertices in the following a matching to secondary vertices is executed. If both test are negative the track is associated always to the first vertex of the vertex list, to the closest vertex in all three dimensions or to the closest vertx in z direction only. The latter option has been defined as the default option. This work flow can be executed multiple times to increase the efficiency. Additionally to each association a quality measure is added. This measure indicates the iteration at which the association has been created and implies a certain purity. \\
In Chapter~\ref{sec:TrackAss} the three options of the third step of the association map have been compared. It has been done in terms of efficiency and purity for preserving signal tracks and subtracting pileup tracks. In most cases, associating the track to the closest vertex in z leads to the best combination of both.  In a next step, the default configuration has been compared to the other pileup subtraction techniques as well as to applying no pileup subtraction. This has been done based on several simulated data sets and individually for charged hadrons, muons and electrons. Except for tracks with a very high \pt{}, the association map lead to the best combination of efficiency and purity.  For those applying no pileup subtraction leads to the best results. Asking only for a high signal efficiency of the pileup subtraction techniques the Jet/MEt approach leads to the best results. When executing the work flow of the association map multiple times the efficiency can be improved to reach the performance of the Jet/MET approach. Doing so, the purity of the association map remains significantly larger than that of the Jet/MET approach.  \\
The memory and time consumption of the different pileup subtraction approaches have been compared in Chapter~\ref{sec:OSP}. Due to the different setting of the subtraction sequence the association map takes the most time while the other two approaches need more virtual memory. This is because the creation of the particular associations take more time compared to the simple selections which are applied at the Muon/Egamma or Jet/MET work flow. On the other hand, the latter two approaches create several intermediate track collection that beed more virtual memory. \\
The effects on high-level objects like the relative isolation or jets have been discussed in Chapter~\ref{sec:OO}. Mostly, the shown results can be explained by the number of tracks that are considered as originating from the signal interaction. For approaches that subtract a rather low number of tracks (applying no pileup subtraction and the Jet/MET approach) the relative isolation is high, the \pt{} response of the reconstructed jets is above one, the corrected \MET{} is high as well while the b-tag efficiency is good. For the othr two approaches (association map and Muon/Egamma approach), which subtract a rather high number of tracks, is the other way around. The relative isolation is smaller, the \pt response is below one, the corrected \MET{} is closer to the expected value and the b-tag efficiency is lower. The latter aspect comes due to the fact that the discrimininant of the b tag has been tuned based on the Jet/MET approach. Therefore, for this efficiency as well as for the corrected jets the performance of the association map is compared to the performance of the Muon/Egamma approach. In both studies the association map leads to better results. To close the gap between the performance of the association map and that of the Jet/MET approach in terms of jet reconstruction and b tagging one track can be associated to two or three vertices. In doing so, the resulting \pt{} is almost as good as that of the Jet/MET approach while the b-jet efficiency is higher. \\
In Chapter~\ref{sec:DC} the outcome of the association mpa based on simulation has been compared to those based on data taken from the CMS experiment in 2012. For this study a sample of events selected by the double muon trigger is selected. Having weighted the dictribution of pileup vertices in most of plots a good agreement is visible. The largest discrepancy is seen for the number of considered signal tracks when associating the track always to the first vertex. Next, the efficiency of association both muons to the same primary vertex is studied. In more than $99\,\%$ of the events the association map preserves the muon pair. Looking at the distribution of the invariant mass of those muons pairs that failed the test, no significant range can be observed. Finally, the performance of the different pileup subtraction techniques has been compared in terms of corrected \MET{}. Also in data, the association map leads to the best results. \\
A small outlook on the upcomming run conditions is shown in Chapter~\ref{sec:HPU}. When increasing the number of pileup interactions and maintaining the size of the interaction region the search of the closest primary vertex is more complicated. Hence, the performances of the default configuration of the assocaition map and of the Muon/Egamma approach worsen significantly. When using the resonstruction algorithms as they were end of 2013 applying no pileup subtraction leads to the best results in terms of track association. The same can be observed for the reconstruction of the jets. On the other hand, in terms of the performance of the correction of the reconstructed missing transverse energy the association map leads clearly to the best resulst for all predictions of the future run conditions. To solve these upcoming problems a few approaches are possible. One part belongs to the properties of the interaction region. The size could be enlarged by reducing the angle under which the proton bunches collide. To further reduce the density of the primary vertices the distribution of the proton in the bunches could be changed. On the other the reconstruction algorithm could and will be changed till the begining of the next run time in 2015 to cope with the harder conditions. \\
The work flow of the association map itself could be improved by a better reconstruction of the secondary vertices. Up to now only a small amount of tracks can be matched to such vertices. If the purity and efficiency in reconstructing secondary vertices is improved also the performance of the association map becomes better. Having done this also the reconstruction of high-level objects would benefit, for instance the b tagging. In the context of a global description of the event it is recommended to create the association map using three iterations. By applying different selection criteria on the quality of the association it is possible to achieve optimal performance in terms of all observable discussed in this work. Having tuned the jet energy correction factors or the discriminant of the b tag an adequate statement coud be made about the impacts on a complete analysis.
