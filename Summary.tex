\chapter{Summary and Outlook \label{sec:SandO}}

In this work a new approach of th esubtraction of tracks coming from pileup vertices has been presented. As introduced in Section~\ref{sec:LHCCMS} until end of 2012 on average about 20 additional proton-proton interactions took place simultaneously to the one signal interaction. These underlying events can have severe impact on th ereconstruction of all objects. To reduce this impact, two techniques have been developed. Both techniques are very similar in that way that they only apply a filter on the trajectory for muons and electrons. For charged hadrons that are not used for the fit of the primary vertices the techniques follow different approaches. In order to obtain better resulsts on terms of track association a new method has been introduced in the scope of this work. It takes advantage of several reconstructed objects like secondary vertices and uses advanced techniques to find the best mathcing primary vertex for a particular track. In Section~\ref{sec:AssMap} the work flow of this association has been explained. It can be divided in three steps. Starting with a test whether the track is used for tha fit of the primary vertices a matching to secondary vertices is executed, secondly. If both test are negative the track is associated always to the first vertex of the vertex list, to the closest vertex in all three dimensions or to the closest vertx in z direction only. The latter option has been defined as the default option. This work flow can be executed mutiple times. Additionally to each association a quality measure is added. This measure indicates the iteration at which the association has been created and implies a certain purity. \\
In Section~\ref{sec:TrackAss} the three options of the third step of the association map have been compared. It has been done in terms of efficiency and purity for preserving signal tracks and subtracting pileup tracks. In most cases, associatin ghte track to the closest vertex in z leads to the best combination of both. In a next step, the default configuration has been compared to the other pileup subtraction techniques as well as to applying no pileup subtraction. This has been done based on several simulated data sets and individually for charged hadrons, muons and electrons. Overall, the association map lead to the best combination of efficiency and purity. \\
In Section~\ref{{sec:OSP} the memory and time consumption of the different pileup subtraction approaches have been compared. Due to the different setting the association map takes the most time while the other two approaches need more virtual memory.