\chapter{Summary and Outlook \label{sec:SandO}}

In this work a new approach of the identification of tracks originating from pileup vertices has been presented. Until end of 2012 on average about 20 additional proton-proton interactions took place simultaneously to the one signal interaction in each bunch crossing. These pileup events can have severe impact on the reconstruction of all physics objects. To reduce this impact, two techniques have been developed by CMS in the past. Both techniques are very similar in that they only apply a selection based on the trajectory for muons and electrons. For charged hadrons that are not used for the fit of the primary vertices the techniques follow different approaches. \\
In order to obtain better results a new method has been introduced in the scope of this work named association map. It takes advantage of several reconstructed objects like secondary vertices, and uses advanced techniques to find the best matching primary vertex for a particular track. The work flow of this association can be divided into three steps using the track weight of the primary vertices, a matching to secondary vertices and a final association where threee different options can be chosen. This work flow can be executed multiple times to increase the efficiency. Additionally to each association a quality measure is established. This measure indicates the iteration at which the association has been created and implies a certain purity. \\
The three options of the third step of the association map have been compared in terms of efficiency and purity for preserving signal tracks and subtracting pileup tracks. In most cases, associating the track to the closest vertex in z leads to the best combination of both.  In a next step, the default configuration has been compared to the other pileup subtraction techniques as well as to applying no pileup subtraction. Except for tracks with a very high \pt{}, the association map leads to the best combination of efficiency and purity. For those, applying no pileup subtraction leads to the best results. By executing the work flow of the association map multiple times the efficiency can be improved to reach the performance of the Jet/MET approach. Doing so, the purity of the association map remains significantly larger. \\
The memory and computing time consumption of the different pileup subtraction approaches have been compared in Chapter~\ref{sec:OSP}. Due to the different setting of the subtraction sequence the association map takes the most time while the other two approaches need more virtual memory. \\
The effects on high-level objects, like the relative isolation or jets, have been shown to be explainable by the number of tracks that are considered as originating from the signal interaction. While the performance based on the association map leads to the best \MET{} calibration, the reconstructied momentum of the jets is too small. This problem can be solbed by multiple associations of one track. The same holds for the b-tag efficiency. \\
The outcome of the association map based on simulation has been compared to those based on data taken from the CMS experiment in 2012. Having weighted the dictribution of pileup vertices a good agreement is visible. Next, the efficiency of associating both muons to the same primary vertex is studied. In more than $99\,\%$ of the events the association map preserves the muon pair. Finally, the performance of the different pileup subtraction techniques has been compared in terms of corrected \MET{}. Also in data, the association map leads to the best results. \\
A small outlook on the performance of the pileup subtraction techniques for the upcomming run conditions is shown. When increasing the number of pileup interactions while maintaining the size of the interaction region the search of the closest primary vertex is more complicated. Hence, the performances of the default configuration of the assocaition map and of the Muon/Egamma approach worsen significantly. When using the resonstruction algorithms as they were end of 2013, applying no pileup subtraction leads to the best results in terms of track association. The same can be observed for the reconstruction of the jets. On the other hand, in terms of the performance of the correction of the reconstructed missing transverse energy the association map leads clearly to the best results for all predictions of the future run conditions. To solve these upcoming problems a few approaches are possible. On the other hand the reconstruction algorithms could and will be changed till the begining of the next run time in 2015 to cope with the harder conditions. \\
The work flow of the association map itself could be improved by a better reconstruction of the secondary vertices. Up to now only a small amount of tracks can be matched to such vertices. In the context of a global description of the event it is recommended to create the association map using three iterations. By applying different selection criteria on the quality of the association it is possible to achieve optimal performance in terms of all observables discussed in this work. Having tuned the jet energy correction factors or the discriminant of the b tag an adequate statement could be made about the impacts on a complete physics analysis.
