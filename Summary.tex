\chapter{Summary and Outlook \label{sec:SandO}}

In this work a new approach to identify tracks originating from pileup vertices has been presented. Until the end of 2012 on average about 20 additional proton-proton interactions took place simultaneously on top of one possible signal interaction in each bunch crossing. These pileup events can have severe impact on the reconstruction of all physics objects. Two established techniques have been developed by CMS in the past t o reduce this impact. Both techniques are similar and only apply a selection based on the trajectory for muons and electrons. For charged hadrons that are not considered in the fit of the primary vertices, the techniques follow different approaches. The present work investigates a more comprehensive approach.

\begin{description}

\item[Implementation] In order to obtain better results a new method has been implemented named association map. It takes advantage of several reconstructed objects like secondary vertices, and uses advanced techniques to find the best matching primary vertex for a particular track. The work flow of this association can be divided into three steps using (a) the track weight of the primary vertices, (b) a matching to secondary vertices and (c) a final association where three different options can be chosen. This work flow can be executed iteratively to increase the efficiency. Additionally to each association, a quality measure is established. This measure indicates the iteration at which the association has been created and implies a certain purity. As consequence, the association map is more flexible than the two established subtraction techniques. Finally, in one event on average almost 500 tracks are associated to about 20 primary vertices.

\item[Optimization] The different options of the association map have been compared in terms of efficiency and purity for preserving signal tracks and subtracting pileup tracks. In most cases, associating the track to the closest vertex in z leads to the best combination of both. Doing so, after on iteration a signal efficiency and a purity of $86\,\%$ is reached. With additional iterations the efficiency can be improved to more than $90\,\%$.

\item[General performance test with respect to established methods] The default configuration has been compared to the other established pileup subtraction techniques as well as to applying no pileup subtraction. Except for tracks with a very high \pt{}, the association map leads to the best combination of efficiency and purity. For high-\pt{} tracks, applying no pileup subtraction leads to the best results. The advantage of the new approach is up to $40\,\%$ in terms of purity. By executing the work flow of the association map multiple times the efficiency can be improved to reach the performance of the Jet/MET approach ($\approx92\,\%$). Doing so, the purity of the association map remains significantly larger (by about $10\,\%$).

\item[Performance in terms of CPU load] The memory and computing time consumption of the different pileup subtraction approaches have been compared. Due to the different setting of the subtraction sequence the association map takes the most time. At a maximum, the creation of the association map needs about 155\unit{ms} for three iterations at current grid computing cores. For one iteration only time time consumption is only one third of that. \\
On the other hand, the other two approaches need more virtual memory. The outcome of the new approach needs a factor of two to three less memory.

\item[Performance on physical objects] The effects on high-level objects, like the relative isolation or jets, have been shown to be explainable by the number of tracks that are considered as originating from the signal interaction. \\
The performance based on the association map leads to the best MET calibration. The difference to the other approaches is in the order of less than $5\,\%$. \\
For the jet reconstruction, without applying momentum corrections the reconstructed momentum behaves more robust for the new approach compared to the established techniques. Applying the correction factors the jet momentum is too small for the default configuration of the new approach. This problem can be solved by multiple associations of one track. \\
The same holds for the b-tag efficiency. One more feature that showed up for the b-jets is that the assignment of tracks from jets or originating from B-hadron decays is more efficient when using the new approach. In this process, the efficiency can be improved from $97\,\%$ to almost $100\,\%$.

\item[Data driven performance] The outcome of the association map based on simulation has been compared to those based on data taken from the CMS experiment in 2012. As physics analysis case Z decays into two muons has been chosen. Having weighted the distribution of pileup vertices a good agreement is visible fro several observables. Next, the efficiency of associating both muons to the same primary vertex is studied. In more than $99\,\%$ of the events the association map preserves the muon pair. \\
Finally, the performance of the different pileup subtraction techniques has been compared in terms of corrected MET. Also in data, the association map leads to the best results. Compared to the current default calibration using the new approach leads to the same results as reducing the average number of pileup interactions by 2 ($\approx10\,\%$).

\item[Outlook on the future perspectives of the LHC] An outlook on the performance of the pileup subtraction techniques for the upcoming run conditions is also drawn. When increasing the number of pileup interactions while maintaining the size of the interaction region the search of the closest primary vertex is more complicated. Hence, the performances of the default configuration of the association map and of the other approaches worsen significantly. If the reconstruction algorithms remain as they were end of 2013, applying no pileup subtraction leads to the best results in terms of track association and jet reconstruction. In contrast, in terms of the correction of the missing transverse energy the association map leads clearly to the best results for all predictions of the future run conditions.

\end{description}


A few approaches are possible to solve these upcoming problems. The reconstruction algorithms will be changed till the beginning of the next run period in 2015 to cope with the harder conditions. The work flow of the association map itself could be improved by a better reconstruction of the secondary vertices. Up to now only a small amount of tracks can be matched to such vertices.
In the context of a global description of the event it is recommended to create the association map using three iterations. By applying different selection criteria on the quality of the association it is possible to achieve optimal performance in terms of all observables discussed in this work. Having tuned the jet energy correction factors or the working point of the b-tagging discriminant an explicit statement could be made about the impacts on a complete physics analysis.
Nevertheless, identifying particles from the signal interactions will remain challenging. Limiting the impact of pileup interactions will play an important role to guarantee a good precision in future analyses.
