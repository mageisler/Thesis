\chapter{On the Agreement between Data and Simulation \label{sec:DCPvalue}}

\begin{figure}[!h]
  \centering
  \includegraphics[width=0.325\textwidth]{DC_App/DataComparison_Real}
  \includegraphics[width=0.325\textwidth]{DC_App/DataComparison_Ratio}
  \includegraphics[width=0.325\textwidth]{DC_App/DataComparison_Difference}
\caption[Comparison between data and simulation: Actual distributions, ratio and difference]{A comparison between data and simulation. Shown are the actual distribution (left hand plot), the ratio (middle plot) and the difference (right hand plot) of data and simulation. On the right hand plot filled circles stand for negative values while empty circles represent positive values. \label{plot:DCPvalue}}
\end{figure}

On the left hand plot in Fig.~\ref{plot:DCPvalue} the distribution of the corrected \MET{} is shown (see also Section~\ref{sec:DCMVDSC}). In order to compare these two distributions the ratio is shown. This is done because the systematic uncertainties are expected to dominate and most of them cancel out in the ratio. Considering the first bin of the distribution in the ratio plot, a discrepancy in the order of four standard deviations can be seen. The uncertainty of the ratio is calculated using Gaussian error propagation. Comparing this to the first bin of the actual distribution on the left hand side, a discrepancy in the order of two standard deviations is expected from intention. The reason for this difference of more than one standard deviation is due to the fact that the ratio of two Gaussian distributions is complicated. While the mean is undefined the variance is infinite~\cite{ratioMail}. Assuming that the statistic uncertainties are dominant one should calculate the difference instead of the ratio. This is done in the right hand plot of Fig.~\ref{plot:DCPvalue}. In the first bin the discrepancy is in the order of two standard deviations. Hence, taking the difference of the two values leads to more intuitive results. Often, the significance of the difference is lower than it is for the ratio. This means that the results of the ratio describe an upper bound of the significance. \\
Although correctly calculating the significance of the ratio is complex, ratio plots are shown in Section~\ref{sec:DC}. In order to make a statement about the compatibility of the two values it should be kept in mind that the actual significances are mostly smaller than the significance appear in the ratio plots based on Gaussian error propagation.
