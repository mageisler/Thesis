\chapter{The Association Map \label{sec:AssMap}}

In the first part of the subtraction of pileup tracks the association map is created. To reach a satisfying association performance also for tracks which are not used for the creation of the primary vertices the technique of the Muon Egamma (see Section~\ref{sec:IntroME}) approach is taken as a role model. For this, in all steps at the search for the best association it is tried to take advantage of the good resolution of the track reconstruction. Additionally, more advanced objects like secondary vertices are taken into account. Furthermore, it is possible to associate one track to more than one vertex. Thus, the efficiency in considering tracks as coming from the signal process is increased. \\
The design of the association map can also be changed. It is possible to assign a set of tracks to one primary vertex or to assign a set of primary vertices to one track. Depending on the analyzer's need both kind of associations have advantages. For example if just the tracks which are considered as coming from the signal vertex are needed tracks should be assigned to primary vertices. On the other hand if two particles are present and their compatibility to the primary vertex need to be checked primary vertices should be assigned to tracks.

\section{Input Collections \label{sec:AMInColl}}

To create the association map a few input collections are mandatory while several others are supplementary. Required are of course a track collection and a set of primary vertices. Furthermore, it needs to be defined which direction of association should be created, tracks to vertices or the other way around. \\ 
It is possible  to enable the association of tracks coming from secondary vertices described in Section~\ref{sec:AMWFSV}. For that, input collections for photon conversions, nuclear interactions, decay of neutral particles and vertices produced by the \textit{inclusive vertex finder} can be set. \\
For the last step it can be chosen which kind of final association should be carried out. Tracks which cannot be matched neither to primary nor to secondary vertices can be assigned to the closest vertex in 3D or in z only. Furthermore, a weight can be set which manipulates the search for the closest vertex (see Section~\ref{sec:AMWFFA}). \\
To allow for more than one associated vertex per track the corresponding input parameter can be changed.

\section{Workflow \label{sec:AMWorkflow}}

A loop is executed on the given track collection. The workflow of finding the best matching primary vertex for a particular track can be divided into three steps. The first step is taking advantage of the track weight stored in the vertices (see Section~\ref{sec:IntroProdPV}) as it is commonly done in the other pileup subtraction techniques, too (see Section~\ref{sec:IntroCurPST}). The second step uses secondary vertices. If a track matches to a secondary vertex it is tried to find the best primary vertex with the information from the secondary vertex. The third step is a search for the vertex closest to the particular track. In the description of this last step a small manipulation of the real distance from the primary vertex to the track is introduced. This modification takes into account the number of tracks that were used to fit the primary vertex. It is also applied at the second step when the closest primary vertex to an estimated incoming particle of a secondary vertex needs to be found. \\
As soon as an association can be made it is stored in the map together with a quality. This quality is defined based on the estimated purity depending on the distance of the association and the iteration at which it has been created. In Figure~\ref{plot:AMTrackStepFrac} the fraction of tracks is shown that are associated at a particular step.

\begin{figure}[!ht]
  \centering
  \includegraphics[width=1.\textwidth]{AM/Workflow}
  \caption[Sketch of the workflow of the association map]{A sketch showing the general workflow of the association map. The input collections are shown in ellipses, the different steps in rectangles. \label{plot:AMWorkSketch}}
\end{figure}

If more than one association should be created this workflow is executed normally at the first iteration. At any following iteration only the third step is applied. This is because one track cannot be matched to more than one primary vertex. Furthermore, a track can only be matched to a primary and a secondary vertex if it is the incoming particle. If this happens the association using the secondary vertex will be based on this incoming track. Hence, the second step in the second iteration can only lead to the same results as the first step of the first iteration or the third step of the second iteration. Due to this, from the second iteration on all association will only be created using the third step. Additionally, matched vertices from previous iterations are removed from the vertex collection so that the same association cannot be created multiple times. This has as a consequence that for events with less reconstructed vertices than required associations the number of created associations is lowered. \\ \\
The here presented configuration is set up based on a simulated data sample. This data sample needs to be a standard sample widely used in CMS. Therefore, a data sample containing \Zz decays into two muons has been chosen. Furthermore, at least one quark-gluon jet from the initial state is simulated. Using this sample all main objects for physics analyses are covered. The simulated center of mass energy was 8\TeV. The number of underlying pileup interactions has been tried to be adopted from real data with a Poissonian distribution with a mean of nearly 20.

\begin{figure}[!ht]
  \centering
  \includegraphics[width=0.55\textwidth]{AM/TrackNum_Vs_Step}
  \caption[Plot of the fraction of tracks to be associated at which step]{The fraction of tracks that were associated at a particular step as a function of the pseudorapidity.\label{plot:AMTrackStepFrac}}
\end{figure}

\subsection{Step 1: Track Weight \label{sec:AMWFTW}}

\begin{figure}[!ht]
  \centering
  \includegraphics[width=0.325\textwidth]{AM/TrackNum_Vs_TrackWeight}
  \includegraphics[width=0.325\textwidth]{AM/TrackWeightAnalysis_Purity}
  \includegraphics[width=0.325\textwidth]{AM/TrackWeightAnalysis_Efficiency}
  \caption[Distribution of the track weight and purity and efficiency vs track weight]{The normalized distribution of the track weight for signal and pileup vertices and the purity and efficiency as a function of the track weight.\label{plot:AMTWdistpureff}}
\end{figure}

At the first step it is checked if a track weight at any vertex is stored for the track.  As it was shown before about $60\,\%$ of the tracks have a track weight at any vertex. The distribution of this track weight can be seen on the left hand side in Figure~\ref{plot:AMTWdistpureff}. \\
The upcoming question is now if the quality of the created association depends on the track weight. For that, the plot on the right hand side of Figure~\ref{plot:AMTWdistpureff} shows the purity and the efficiency as a function of the track weight. The efficiency is here defined as the fraction of those simulated signal tracks that have been reconstructed and correctly assigned to the signal vertex
\begin{equation}
\textrm{efficiency} = \frac{\textrm{simulated as signal track and reconstructed \textbf{and assigned to signal vertex}}}{\textrm{simulated as signal track and reconstructed}}
\label{eq:Efficiency}
\end{equation}
while the purity is defined as the fraction of those reconstructed tracks which are considered as coming from the signal vertex and can also be matched to a simulated signal track:
\begin{equation}
\textrm{purity} = \frac{\textrm{reco track assigned to the signal vertex \textbf{and matched to a sim signal track}}}{\textrm{reco track assigned to the signal vertex}}
\label{eq:Purity}
\end{equation}
While the efficiency accounts only for association issues the purity also accounts for reconstructed tracks that haven't been simulated. The reason for this is that the quality assignment presented in Section~\ref{sec:AMWFQD} will be based on this purity. This purity needs to be based on all reconstructed tracks and not only on those which can be matched to a simulated one. \\ 
As it will be shown the values for both, purity and efficiency, are better than the values reached by the following steps (see Sections~\ref{sec:AMWFSV} and~\ref{sec:AMWFFA} ). Moreover, an efficiency of nearly one is reached for all ranges of the track weight. That means, that if a simulated signal track can be matched to a reconstructed track that has a track weight greater than 0 at any primary vertex this primary vertex is always the first one. Therefore, no cut on the track weight is applied. In other words, as soon as the stored track weight is greater than 0 the track is assigned to the particular vertex and put into the association map.

\subsection{Looking for the Closest Vertex\label{sec:AMWFCV}}

In all following steps the most probable primary vertex has to be found. It is done by looking for the closest vertex to a given track. This can be done in two different ways. The first one is to look for the closest vertex in all three dimensions. The other one is to look in z only because of the largest separation of the primary vertices along this axis. To look for the closest vertex in all three dimension a trajectory needs to be build. It describes the tracks curvature through the interaction region. Then for every primary vertex the minimum distance to this trajectory is calculated. To look for the closest vertex in z the position of closest approach to the beam spot is already saved as the tracks vertex. Therefore, no creation of a trajectory is needed. To get the closest primary vertex the distance between the z position of the tracks vertex and the primary vertex is taken. In both approaches this real distance gets modified in that way that the number of tracks used to fit the vertex is taken into account. Imagining a track which is reconstructed as passing between two primary vertices right in the middle. Vertex 1 has been fitted based on 100 tracks while vertex 2 was fitted based on 5 tracks only. In this case it is more likely that the track comes also from vertex 1 instead of vertex 2. In order to take situations like this into account this modification has been invented. 

\begin{displaymath}
\textrm{modified distance} = \textrm{real distance} - \textrm{weight} \cdot n_{tracks}
\end{displaymath}
Here, weight stands for a certain weight with which the number of tracks is multiplied. This weight has been varied between 0. and 0.1 to find the best value. A weight of 0 stands for no modification. Furthermore it is tested whether the actual number of tracks should be used or the square root. The comparison which of these weights leads to the best results is shown in Figure~\ref{plot:AMWFCV3DFalse} and Figure~\ref{plot:AMWFCV3DTrue} for looking in all three dimensions and in Figure~\ref{plot:AMWFCVZFalse} and Figure~\ref{plot:AMWFCVZTrue} for looking along the z-axis only, respectively. It is always shown as a function of the distance in the x-y-plane from the beam axis to the position of closest approach of the track $d_{xy}$.

\begin{figure}[!ht]
  \centering
  \includegraphics[width=0.325\textwidth]{AM/FinalAssociation_Eff_3_False}
  \includegraphics[width=0.325\textwidth]{AM/FinalAssociation_Pur_3_False}
  \includegraphics[width=0.325\textwidth]{AM/FinalAssociation_Pro_3_False}
  \caption[Efficiency, purity and their product for different track weights with the actual number of tracks for the search in three dimensions.]{A comparison of the efficiency (left hand plot), purity (middle plot) and the product (right hand plot) of both for the search for the closest vertex in all three dimensions for different track weights. It is done with the actual number of tracks.\label{plot:AMWFCV3DFalse}}
\end{figure}

\begin{figure}[!ht]
  \centering
  \includegraphics[width=0.325\textwidth]{AM/FinalAssociation_Eff_3_True}
  \includegraphics[width=0.325\textwidth]{AM/FinalAssociation_Pur_3_True}
  \includegraphics[width=0.325\textwidth]{AM/FinalAssociation_Pro_3_True}
  \caption[Efficiency, purity and their product for different track weights with the square root or the number of tracks for the search in three dimensions.]{A comparison of the efficiency (left hand plot), purity (middle plot) and the product (right hand plot) of both for the search for the closest vertex in all three dimensions for different track weights. It is done with the square root of the  number of tracks.\label{plot:AMWFCV3DTrue}}
\end{figure}

\begin{figure}[!ht]
  \centering
  \includegraphics[width=0.325\textwidth]{AM/FinalAssociation_Eff_z_False}
  \includegraphics[width=0.325\textwidth]{AM/FinalAssociation_Pur_z_False}
  \includegraphics[width=0.325\textwidth]{AM/FinalAssociation_Pro_z_False}
  \caption[Efficiency, purity and their product for different track weights with the actual number of tracks for the search along z-axis only.]{A comparison of the efficiency (left hand plot), purity (middle plot) and the product (right hand plot) of both for the search for the closest vertex along the z-axis only for different track weights. It is done with the actual number of tracks.\label{plot:AMWFCVZFalse}}
\end{figure}

\begin{figure}[!ht]
  \centering
  \includegraphics[width=0.325\textwidth]{AM/FinalAssociation_Eff_z_True}
  \includegraphics[width=0.325\textwidth]{AM/FinalAssociation_Pur_z_True}
  \includegraphics[width=0.325\textwidth]{AM/FinalAssociation_Pro_z_True}
  \caption[Efficiency, purity and their product for different track weights with the square root or the number of tracks for the search along z-axis only.]{A comparison of the efficiency (left hand plot), purity (middle plot) and the product (right hand plot) of both for the search for the closest vertex along the z-axis only for different track weights. It is done with the square root of the  number of tracks.\label{plot:AMWFCVZTrue}}
\end{figure}

While there are some combinations that lead to significantly worse results for either the efficiency or the purity some settings lead to very similar results. Based on the combination of a good efficiency and purity the following modifications for the search in all three dimensions and along the z axis only the following combinations have been chosen:
\begin{eqnarray}
\textrm{3D}: \textrm{modified distance} & = & \textrm{real distance} - 0.0001 \cdot n_{tracks} \label{eqn:AMWFCVweight3} \\
\textrm{z}: \textrm{modified distance} & = & \textrm{real distance} - 0.01 \cdot \sqrt{n_{tracks}} \label{eqn:AMWFCVweightz}
\end{eqnarray}

\subsection{Step 2: Secondary Vertices\label{sec:AMWFSV}}

If the track does not have a track weight stored at any primary vertex it is check if the track matches to a secondary vertex. For this there are several possible collections of secondary vertices that can be used. In Figure~\ref{plot:AMWFSVfrac} the number of tracks per events is shown that can be matched to a secondary vertex. The peak at around 1\cm to 7\cm can be explained by the beam pipe and the first layer of the pixel tracker. At these positions interactions like photon conversions take place. The first peak comes due to the reconstructed secondary vertices from b jets. The general workflow of the association is always the same. If the track matches to a certain secondary vertex the position of the vertex is taken. Furthermore, the momentum of the incoming particle is estimated. With these two values the most probable primary vertex is tried to find. For this, there are two possible ways like searching in z only or in all three dimension.\\
Moreover, for every collection of secondary vertices it is checked if a filter on the secondary vertex would improve the performance of the association in terms of efficiency or purity. 

\begin{figure}[!ht]
  \centering
  \includegraphics[width=0.55\textwidth]{AM/ReassosTest_TrackNum_Vs_dxy}
  \caption[Number of tracks that can be matched to a secondary vertex]{The number of tracks per event that can be matched to a secondary vertex as a function of the distance in the x-y-plane between the beam axis and the point of closet approach of the track.\label{plot:AMWFSVfrac}}
\end{figure}

\subsubsection{Photon Conversion \label{sec:AMWFSVpc}}
If a collection of photon conversions is given as an input it is checked if the track matches to any conversion. For that there is already a suitable member function existing in the tools for conversions. More information about the reconstruction of photon conversions can be found in reference~\citen{GiordanoConversion}, \citen{CMS-PAS-EGM-10-005} or \citen{CMS-PAS-TRK-10-003}. \\
For every conversion it is checked if the given track was used to create the particular conversion. If so, the position of the conversion and the estimated momentum are taken. Both values have been refitted using the pair of tracks that were identified as coming from the conversion. With this position and the momentum a track is created which describes the trajectory of a neutral particle. The closest primary vertex to this trajectory is then chosen as that vertex to which the initial track is assigned to.  This approach has as a consequence that both tracks that match to one conversion are assigned to the same primary vertex. \\
Based on the results of Section~\ref{sec:AMWFCV} it is looked for the best way to associate the outcoming particles from a conversion. For that the efficiency and purity for four different search options are compared: 
\begin{enumerate}
\item looking for the closest vertex to the reconstructed secondary particle in all three dimension.
\item looking for the closest vertex to the reconstructed secondary particle along the z-axis only.
\item looking for the closest vertex to the estimated primary particle in all three dimension.
\item looking for the closest vertex to the estimated primary particle along the z-axis only..
\end{enumerate}
The results are shown in Figure~\ref{plot:AMWFSVpcEffAndPurSO}. Especially for the efficiency, taking the closest primary vertex to the estimated photon in all three dimensions leads to the best results. Therefore, this option is used for photon conversions.

\begin{figure}[!ht]
  \centering
  \includegraphics[width=0.325\textwidth]{AM/ReassosTest_conv_Efficiency_Vs_Val}
  \includegraphics[width=0.325\textwidth]{AM/ReassosTest_conv_Purity_Vs_Val}
  \includegraphics[width=0.325\textwidth]{AM/ReassosTest_conv_Product_Vs_Val}
  \caption[Efficiency, purity and their product of the association with photon conversions for different search options as a function of distance to the beam axis]{The efficiency (left hand plot), purity (middle plot) and the product of both (right hand plot) for the track association matched to photon conversions. The distance to the beam axis is shown on the x-axis.\label{plot:AMWFSVpcEffAndPurSO}}
\end{figure}

The next question is, if the quality of the reconstructed conversion has a big impact on the performance of the association. For this, a filter on the conversions is applied asking for \textit{highPurity}. To estimate the impact of this filter on the performance of the conversion reconstruction the efficiency and purity are calculated. For this the seeds of a reconstructed conversion are matched to simulated conversions. The results can be seen in Figure~\ref{plot:AMWFSVpcEffAndPurDC}. They are as expected, the average difference in efficiency is in the order or $10\,\%$ while the purity changes only minimal except for one bin only.

\begin{figure}[!ht]
  \centering
  \includegraphics[width=0.325\textwidth]{AM/SecondaryValidator_conv_Efficiency}
  \includegraphics[width=0.325\textwidth]{AM/SecondaryValidator_conv_Purity}
  \includegraphics[width=0.325\textwidth]{AM/UnCleanedComparison_conv_TrackNum_Vs_Val}
  \caption[Purity and efficiency of the reconstructed photon conversions as a function of distance to the beam axis and number of matched tracks]{The efficiency (left hand plot) and purity (middle plot) for the reconstructed photon conversions as a function of distance to the beam axis. On the right hand plot the number of matched tracks per event is shown. Filtered stands for high purity conversions. \label{plot:AMWFSVpcEffAndPurDC}}
\end{figure}

The impact on the performance of the association are shown in Figure~\ref{plot:AMWFSVpcEffAndPurRT}. Comparing the results for the efficiency and purity it is found out that the results for the unfiltered collection of photon conversions are slightly better. For this study, all tracks which were matched to a conversions from the unfiltered collection but not matched to a filtered conversions were associated to a primary vertex using step 3 (see Section~\ref{sec:AMWFFA})

\begin{figure}[!ht]
  \centering
  \includegraphics[width=0.325\textwidth]{AM/UnCleanedComparison_conv_Efficiency_Vs_Val}
  \includegraphics[width=0.325\textwidth]{AM/UnCleanedComparison_conv_Purity_Vs_Val}
  \includegraphics[width=0.325\textwidth]{AM/UnCleanedComparison_conv_Product_Vs_Val}
  \caption[Efficiency, purity and their product for the association with photon conversions a function of distance to the beam axis]{The purity (left hand plot), purity (middle plot) and the product of both (right hand plot) for the association with photon conversions a function of distance to the beam axis. Filtered stands for high purity conversions.\label{plot:AMWFSVpcEffAndPurRT}}
\end{figure}

Finally, it is checked if the association created with the help of the reconstructed conversion is always better than the possible association created at the final step (see Section~\ref{sec:AMWFFA}). It is plausible that for tracks which have no missed hit at the beginning of the track are more likely to be a primary track. For those the association at the final step should be better. The comparison can be found in Figure~\ref{plot:AMWFSVpcEffAndPurVsFA}. Taking the information from the reconstructed secondary vertex leads to a slightly better performance compared to the final step. Therefore, no filter on the number of missed hits is set.

\begin{figure}[!ht]
  \centering
  \includegraphics[width=0.325\textwidth]{AM/SecondaryComparison_conv_Efficiency_Vs_nMissHits}
  \includegraphics[width=0.325\textwidth]{AM/SecondaryComparison_conv_Purity_Vs_nMissHits}
  \includegraphics[width=0.325\textwidth]{AM/SecondaryComparison_conv_Product_Vs_nMissHits}
  \caption[Efficiency, purity and their product for  the association with photon conversions vs final association as a function of missed inner hits]{The purity (left hand plot), purity (middle plot) and the product of both (right hand plot) between the association using the secondary information of photon conversions and the final step as a function of the number of missed inner hits.\label{plot:AMWFSVpcEffAndPurVsFA}}
\end{figure}

\subsubsection{Neutral Particle Decays \label{sec:AMWFSVnd}}
To take tracks coming from decays of neutral particles into account it is possible to assign a collection of \PKzS{} or \PgL{} as an input. By definition of the producer all neutral particle decays have always two outgoing, oppositely charged tracks. More information about the reconstruction of neutral particle decays can be found in reference~\citen{v0paper}. \\
To check if a reconstructed track matches to such a decay a loop is carried out over all neutral decays from the given collections. The daughters of a particular decay can be accessed and need to be casted dynamically into a track to compare it to the respective reconstructed track. Based on the summed momentum of the two daughters and the position of the decay the trajectory of the incoming particle is tried to be reconstructed. \\
With this trajectory the most likely primary vertex can be found.  For that the efficiency and purity for four different search options are compared: to look for the closest vertex in all three dimension and in z only for the estimated primary particle and for the reconstructed secondary particle, respectively. The results are shown in Figure~\ref{plot:AMWFSVkdecEffAndPurSO} and Figure~\ref{plot:AMWFSVldecEffAndPurSO}. Again especially for the efficiency taking the closest primary vertex to the estimated neutral particle in all three dimensions (labeled "Primary 3D" in the plots) leads to the best results. Therefore, this option is also used for neutral particle decays.



\begin{figure}[!ht]
  \centering
  \includegraphics[width=0.325\textwidth]{AM/ReassosTest_kdec_Efficiency_Vs_Val}
  \includegraphics[width=0.325\textwidth]{AM/ReassosTest_kdec_Purity_Vs_Val}
  \includegraphics[width=0.325\textwidth]{AM/ReassosTest_kdec_Product_Vs_Val}
  \caption[Efficiency, purity and their product of the association with \PKzS{} decays for different search options as a function of distance to the beam axis]{The efficiency (left hand plot), purity (middle plot) and the product of both (right hand plot) for the track association matched to \PKzS{} decayss. The distance to the beam axis is shown on the x-axis.\label{plot:AMWFSVkdecEffAndPurSO}}
\end{figure}
\begin{figure}[!ht]
  \centering
  \includegraphics[width=0.325\textwidth]{AM/ReassosTest_ldec_Efficiency_Vs_Val}
  \includegraphics[width=0.325\textwidth]{AM/ReassosTest_ldec_Purity_Vs_Val}
  \includegraphics[width=0.325\textwidth]{AM/ReassosTest_ldec_Product_Vs_Val}
  \caption[Efficiency, purity and their product of the association with \PgL{} decays for different search options as a function of distance to the beam axis]{The efficiency (left hand plot), purity (middle plot) and the product of both (right hand plot) for the track association matched to \PgL{} decays. The distance to the beam axis is shown on the x-axis.\label{plot:AMWFSVldecEffAndPurSO}}
\end{figure}

Again it is checked if a filter needs to be set on the quality of the reconstructed decays. For that the following cuts can be motivated for the reconstructed decays of \PKzS{} (\PgL{}):
\begin{itemize}
\item the normalized $\chi^{2}$ of the fitted decay vertex smaller than 7.
\item the invariant mass of the combined momentum of the daughters should be 9\MeVcc{} (4\MeVcc{}) around the nominal \PKzS{} (\PgL{}) mass.
\item the distance from the position of the beam axis to the decay position divided by both errors combined should be greater than 25. (\PKzS{}) or 27.  (\PgL{}).
\end{itemize}
To estimate the impact of this filter on the performance of the neutral particle decay reconstruction the efficiency and purity are calculated. For this for all simulated \PKzS{} and \PgL{} it is checked if they decay into two particles whose tracks can be reconstructed. This means that they leave at least three hits in the tracker and have a transverse momentum greater than 1\GeV. Decays fulfilling these conditions are then matched to reconstructed neutral particle decay if they are lying within a range of five standard deviations of the reconstruction uncertainty. The results can be seen in Figure~\ref{plot:AMWFSVkdecEffAndPurDC} and Figure~\ref{plot:AMWFSVldecEffAndPurDC}.

\begin{figure}[!ht]
  \centering
  \includegraphics[width=0.325\textwidth]{AM/SecondaryValidator_kdec_Efficiency}
  \includegraphics[width=0.325\textwidth]{AM/SecondaryValidator_kdec_Purity}
  \includegraphics[width=0.325\textwidth]{AM/UnCleanedComparison_kdec_TrackNum_Vs_Val}
  \caption[Purity and efficiency of the reconstructed \PKzS{} decays as a function of distance to the beam axis and number of matched tracks]{The efficiency (left hand plot) and purity (middle plot) for the reconstructed \PKzS{} decays as a function of distance to the beam axis. On the right hand plot the number of matched tracks per event is shown. \label{plot:AMWFSVkdecEffAndPurDC}}
\end{figure}
\begin{figure}[!ht]
  \centering
  \includegraphics[width=0.325\textwidth]{AM/SecondaryValidator_ldec_Efficiency}
  \includegraphics[width=0.325\textwidth]{AM/SecondaryValidator_ldec_Purity}
  \includegraphics[width=0.325\textwidth]{AM/UnCleanedComparison_ldec_TrackNum_Vs_Val}
  \caption[Purity and efficiency of the reconstructed \PgL{} decays as a function of distance to the beam axis and number of matched tracks]{The efficiency (left hand plot) and purity (middle plot) for the reconstructed \PgL{} decays as a function of distance to the beam axis. On the right hand plot the number of matched tracks per event is shown. \label{plot:AMWFSVldecEffAndPurDC}}
\end{figure}

The impact on the performance of the association is shown in Figure~\ref{plot:AMWFSVkdecEffAndPurRT} and Figure~\ref{plot:AMWFSVldecEffAndPurRT} . Comparing the results for the efficiency and purity it is found that the results for the unfiltered collection of neutral particle decays are slightly better. For this, all tracks, which were matched to a neutral particle decay from the unfiltered collection and not matched to a filtered decays, were associated to a primary vertex using step 3 (see Section~\ref{sec:AMWFFA})

\begin{figure}[!ht]
  \centering
  \includegraphics[width=0.325\textwidth]{AM/UnCleanedComparison_kdec_Efficiency_Vs_Val}
  \includegraphics[width=0.325\textwidth]{AM/UnCleanedComparison_kdec_Purity_Vs_Val}
  \includegraphics[width=0.325\textwidth]{AM/UnCleanedComparison_kdec_Product_Vs_Val}
  \caption[Efficiency, purity and their product for the association with \PKzS{} decays a function of distance to the beam axis]{The purity (left hand plot), purity (middle plot) and the product of both (right hand plot) for the association with \PKzS{} decays a function of distance to the beam axis. \label{plot:AMWFSVkdecEffAndPurRT}}
\end{figure}
\begin{figure}[!ht]
  \centering
  \includegraphics[width=0.325\textwidth]{AM/UnCleanedComparison_ldec_Efficiency_Vs_Val}
  \includegraphics[width=0.325\textwidth]{AM/UnCleanedComparison_ldec_Purity_Vs_Val}
  \includegraphics[width=0.325\textwidth]{AM/UnCleanedComparison_ldec_Product_Vs_Val}
  \caption[Efficiency, purity and their product for the association with \PgL{} decays a function of distance to the beam axis]{The purity (left hand plot), purity (middle plot) and the product of both (right hand plot) for the association with \PgL{} decays a function of distance to the beam axis. \label{plot:AMWFSVldecEffAndPurRT}}
\end{figure}

Finally, the association was checked using decays of neutral particles to see if there was an improvement with the final step. Figure~\ref{plot:AMWFSVkdecEffAndPurVsFA} and Figure~\ref{plot:AMWFSVldecEffAndPurVsFA} show that over the whole range of number of missed hits, the association using the information of the neutral particle decays leads to better results. Therefore, no filter on the number of missed inner hits is used.

\begin{figure}[!ht]
  \centering
  \includegraphics[width=0.325\textwidth]{AM/SecondaryComparison_kdec_Efficiency_Vs_nMissHits}
  \includegraphics[width=0.325\textwidth]{AM/SecondaryComparison_kdec_Purity_Vs_nMissHits}
  \includegraphics[width=0.325\textwidth]{AM/SecondaryComparison_kdec_Product_Vs_nMissHits}
  \caption[Efficiency, purity and their product for the association with \PKzS{} decays vs final association as a function of missed inner hits]{The purity (left hand plot), purity (middle plot) and the product of both (right hand plot) between the association using the secondary information of \PKzS{} decays and the final step as a function of the number of missed inner hits.\label{plot:AMWFSVkdecEffAndPurVsFA}}
\end{figure}
\begin{figure}[!ht]
  \centering
  \includegraphics[width=0.325\textwidth]{AM/SecondaryComparison_ldec_Efficiency_Vs_nMissHits}
  \includegraphics[width=0.325\textwidth]{AM/SecondaryComparison_ldec_Purity_Vs_nMissHits}
  \includegraphics[width=0.325\textwidth]{AM/SecondaryComparison_ldec_Product_Vs_nMissHits}
  \caption[Efficiency, purity and their product for the association with \PgL{} decays vs final association as a function of missed inner hits]{The purity (left hand plot), purity (middle plot) and the product of both (right hand plot) between the association using the secondary information of \PgL{} decays and the final step as a function of the number of missed inner hits.\label{plot:AMWFSVldecEffAndPurVsFA}}
\end{figure}

\subsubsection{Nuclear Interaction \label{sec:AMWFSVni}}

Another possible collection of secondary vertices are the nuclear interactions. More information about the reconstruction of nuclear interactions can be found in reference~\citen{CMS-PAS-TRK-10-003}. \\
To test if a track comes from such an interaction the track weight can be used because these secondary vertices are fitted in an analogous manner to primary ones.  The search for the most likely primary vertex for matching tracks is different to the previous secondary vertices. It is possible that the incoming particle of the nuclear interaction is charged. Hence, the belonging track could be reconstructed, too. Therefore, it is checked if the incoming track has been reconstructed first. If so, it is tested to see if this incoming track has a track weight greater than 0 at one of the primary vertices. If not the closest primary vertex to the incoming track is used. If the incoming track has not been reconstructed, the position of the nuclear interaction is taken as well as the fitted momentum. Based on these, a trajectory of a neutral particle is build. With this trajectory, the closest primary vertex is found. If no incoming particle is present the trajectory is created based on the information of the secondary track. To find the best way to identify the closest vertex four different kind of searching are compared: 
\begin{enumerate}
\item using the technique described above looking for the closest vertex in all three dimensions.
\item using the technique described above looking for the closest vertex along the z-axis only.
\item using the closest vertex in all three dimensions to the secondary track always. 
\item using the closest vertex along the z-axis only to the secondary track always. 
\end{enumerate}
The results are shown in Figure~\ref{plot:AMWFSVnuciEffAndPurSO}. Taking the closest primary vertex to the estimated incoming particle in all three dimensions leads to the best combination of purity and efficiency. Therefore, this option is also used for nuclear interactions.

\begin{figure}[!ht]
  \centering
  \includegraphics[width=0.325\textwidth]{AM/ReassosTest_nuci_Efficiency_Vs_Val}
  \includegraphics[width=0.325\textwidth]{AM/ReassosTest_nuci_Purity_Vs_Val}
  \includegraphics[width=0.325\textwidth]{AM/ReassosTest_nuci_Product_Vs_Val}
  \caption[Efficiency, purity and their product of the association with nuclear interactions for different search options as a function of distance to the beam axis]{The efficiency (left hand plot), purity (middle plot) and the product of both (right hand plot) for the track association matched to nuclear interactions. The distance to the beam axis is shown on the x-axis.\label{plot:AMWFSVnuciEffAndPurSO}}
\end{figure}

The next test is to determine whether or not a filter should be set on the quality of the nuclear interactions. For that, the following cuts are motivated:
\begin{itemize}
\item the normalized $\chi^{2}$ of the fitted decay vertex smaller than 2.
\item the number of outgoing tracks which were used to fit the interaction vertex should be greater than 2.
\item the distance from the position of the beam axis to the interaction position should be greater than 3\cm.
\end{itemize}
To estimate the impact of this filter on the performance of the nuclear interaction reconstruction the efficiency and purity are calculated. For this all simulated secondary vertices are checked to see if at least two outgoing particles can be reconstructed. This means that these tracks leave at least three hits in the tracker and have a transverse momentum greater than 1\GeV. In the following step, if a reconstructed nuclear interaction lies within a range of five standard deviations of the reconstruction uncertainty to a simulated one the two are matched. The results can be seen in Figure~\ref{plot:AMWFSVnuciEffAndPurDC}. 

\begin{figure}[!ht]
  \centering
  \includegraphics[width=0.325\textwidth]{AM/SecondaryValidator_nuci_Efficiency}
  \includegraphics[width=0.325\textwidth]{AM/SecondaryValidator_nuci_Purity}
  \includegraphics[width=0.325\textwidth]{AM/UnCleanedComparison_nuci_TrackNum_Vs_Val}
  \caption[Purity and efficiency of the reconstructed nuclear interactions as a function of distance to the beam axis and number of matched tracks]{The efficiency (left hand plot) and purity (middle plot) for the reconstructed nuclear interactions as a function of distance to the beam axis. On the right hand plot the number of matched tracks per event is shown. \label{plot:AMWFSVnuciEffAndPurDC}}
\end{figure}

The impact on the performance of the association is shown in Figure~\ref{plot:AMWFSVnuciEffAndPurRT}. Comparing the results for the efficiency and purity it is seen that the unfiltered collection for the nuclear interactions is slightly better. For this study, all tracks which were matched to a nuclear interaction from the unfiltered collection, but not matched to a filtered nuclear interaction, were associated to a primary vertex using step 3 (see Section~\ref{sec:AMWFFA})

\begin{figure}[!ht]
  \centering
  \includegraphics[width=0.325\textwidth]{AM/UnCleanedComparison_nuci_Efficiency_Vs_Val}
  \includegraphics[width=0.325\textwidth]{AM/UnCleanedComparison_nuci_Purity_Vs_Val}
  \includegraphics[width=0.325\textwidth]{AM/UnCleanedComparison_nuci_Product_Vs_Val}
  \caption[Efficiency, purity and their product for the association with nuclear interactions a function of distance to the beam axis]{The purity (left hand plot), purity (middle plot) and the product of both (right hand plot) for the association with nuclear interactions a function of distance to the beam axis. \label{plot:AMWFSVnuciEffAndPurRT}}
\end{figure}

Finally, it is demonstrated that the association using the information of the reconstructed nuclear interaction is better than using the final association. This comparison can be seen in Figure~\ref{plot:AMWFSVnuciEffAndPurVsFA}. As illustrated, using the nuclear interactions leads to slightly better results over the whole range of number of missed inner hits.

\begin{figure}[!ht]
  \centering
  \includegraphics[width=0.325\textwidth]{AM/SecondaryComparison_nuci_Efficiency_Vs_nMissHits}
  \includegraphics[width=0.325\textwidth]{AM/SecondaryComparison_nuci_Purity_Vs_nMissHits}
  \includegraphics[width=0.325\textwidth]{AM/SecondaryComparison_nuci_Product_Vs_nMissHits}
  \caption[Efficiency, purity and their product for the association with nuclear interactions vs final association as a function of missed inner hits]{The purity (left hand plot), purity (middle plot) and the product of both (right hand plot) between the association using the secondary information of nuclear interactions and the final step as a function of the number of missed inner hits.\label{plot:AMWFSVnuciEffAndPurVsFA}}
\end{figure}

\subsubsection{Inclusive Vertex Finder \label{sec:AMWFSVivf}}

A last possible input collection for secondary vertices is a collection for so-called \textit{inclusive vertices}. The main purpose of this collection is to reconstruct vertices of jets as they are created by \PB{} decays. More information about these secondaries can be found in reference~\citen{ivfPaper}. \\
To check if a track comes from such an inclusive vertex, it is again possible to ask if a track weight has been stored in the secondary. First, the best way to look for the most suitable primary vertex needs to be found for tracks which were matched to inclusive vertices. With the information from the inclusive vertex, a trajectory of the estimated incoming particle can be created. With this trajectory, it is possible to look for the closest primary vertex in all three dimensions or only in z direction. Moreover, it is tested to see if an association using only the trajectory of the secondary particle leads to better results. Efficiency and purity of this comparison are shown in Figure~\ref{plot:AMWFSVinvfEffAndPurSO}. Again it is shown that using the estimated primary particle and looking for the closest vertex in all three dimensions leads to the best results. 

\begin{figure}[!ht]
  \centering
  \includegraphics[width=0.325\textwidth]{AM/ReassosTest_invf_Efficiency_Vs_Val}
  \includegraphics[width=0.325\textwidth]{AM/ReassosTest_invf_Purity_Vs_Val}
  \includegraphics[width=0.325\textwidth]{AM/ReassosTest_invf_Product_Vs_Val}
  \caption[Efficiency, purity and their product of the association with inclusive vertices for different search options as a function of distance to the beam axis]{The efficiency (left hand plot), purity (middle plot) and the product of both (right hand plot) for the track association matched to inclusive vertices. The distance to the beam axis is shown on the x-axis.\label{plot:AMWFSVinvfEffAndPurSO}}
\end{figure}

Next, it is tested if a filter on the inclusive vertices could lead to better results. For this, the following cuts are motivated:
\begin{itemize}
\item a normalized $\chi^{2}$ smaller than 2.
\item the number of tracks which were used to fit the inclusive vertex should be greater than 2.
\item the distance from the beam axis to the position of the inclusive vertex divided by the error should be greater than 10.
\end{itemize}

In Figure~\ref{plot:AMWFSVinvfEffAndPurDC} the impact of these cuts on the inclusive vertices is shown. For these calculations, all simulated secondary vertices were taken into account which have at least two outgoing tracks that can be reconstructed similar to the conditions for the nuclear interactions. A simulated secondary is considered to be reconstructed if a reconstructed vertex lies within a range of five standard deviations. 

\begin{figure}[!ht]
  \centering
  \includegraphics[width=0.325\textwidth]{AM/SecondaryValidator_invf_Efficiency}
  \includegraphics[width=0.325\textwidth]{AM/SecondaryValidator_invf_Purity}
  \includegraphics[width=0.325\textwidth]{AM/UnCleanedComparison_invf_TrackNum_Vs_Val}
  \caption[Purity and efficiency of the reconstructed inclusive vertices as a function of distance to the beam axis and number of matched tracks]{The efficiency (left hand plot) and purity (middle plot) for the reconstructed inclusive vertices as a function of distance to the beam axis. On the right hand plot the number of matched tracks per event is shown. \label{plot:AMWFSVinvfEffAndPurDC}}
\end{figure}

The impact on the association of track that were matched to an inclusive vertex can be seen in Figure~\ref{plot:AMWFSVinvfEffAndPurRT}. Although applying the filter leads to a higher purity in combination with the efficiency it is slightly worse. Therefore, in the following no filter is applied. Again, tracks that were matched to a inclusive vertex which does not pass the filter is associated using step 3 (see Section~\ref{sec:AMWFFA}).

\begin{figure}[!ht]
  \centering
  \includegraphics[width=0.325\textwidth]{AM/UnCleanedComparison_invf_Efficiency_Vs_Val}
  \includegraphics[width=0.325\textwidth]{AM/UnCleanedComparison_invf_Purity_Vs_Val}
  \includegraphics[width=0.325\textwidth]{AM/UnCleanedComparison_invf_Product_Vs_Val}
  \caption[Efficiency, purity and their product for the association with inclusive vertices a function of distance to the beam axis]{The purity (left hand plot), purity (middle plot) and the product of both (right hand plot) for the association with inclusive vertices a function of distance to the beam axis. \label{plot:AMWFSVinvfEffAndPurRT}}
\end{figure}

Finally, the efficiency and purity of the association using the secondary information is compared with the third step as a function of the number of missed inner hits.It can be seen from Figure~\ref{plot:AMWFSVinvfEffAndPurVsFA} that the association with the information of the inclusive vertices (labeled "Secondary") leads mostly to better results.

\begin{figure}[!ht]
  \centering
  \includegraphics[width=0.325\textwidth]{AM/SecondaryComparison_invf_Efficiency_Vs_nMissHits}
  \includegraphics[width=0.325\textwidth]{AM/SecondaryComparison_invf_Purity_Vs_nMissHits}
  \includegraphics[width=0.325\textwidth]{AM/SecondaryComparison_invf_Product_Vs_nMissHits}
  \caption[Efficiency, purity and their product for the association with inclusive vertices vs final association as a function of missed inner hits]{The purity (left hand plot), purity (middle plot) and the product of both (right hand plot) between the association using the secondary information of inclusive vertices and the final step as a function of the number of missed inner hits.\label{plot:AMWFSVinvfEffAndPurVsFA}}
\end{figure}


\subsection{Step 3: Final association\label{sec:AMWFFA}}

This step is executed if the track could not be associated to any primary vertex. As it can be seen in Figure~\ref{plot:AMTrackStepFrac} about $40\,\%$ to $50\,\%$ of all tracks are associated in this step. There are three different possibilities to associate the track. \\
The first one is to assign the track always to the first vertex of the vertex collection. This option has as a consequence that about $40\,\%$ of all tracks will be associated to the first primary vertex. Therefore, it leads to a very low purity. Here, it needs to be remembered that after an iteration the associated vertex is removed from the vertex collection. Hence, if this method is chosen in the second iteration the track will be associated to the second vertex from the collection. The definition of the gradation of the vertices can be found in Section~\ref{sec:IntroSigVer}. \\
The second and the third option to find the most likely primary vertex have already been introduced in Section~\ref{sec:AMWFCV}. The search for the closest vertex can be done in all three dimensions or in z only, both with their particular modification as given in Equation~\ref{eqn:AMWFCVweight3} and~\ref{eqn:AMWFCVweightz}.

\begin{figure}[!ht]
  \centering
  \includegraphics[width=0.325\textwidth]{AM/FinalAssociation_Eff_Third}
  \includegraphics[width=0.325\textwidth]{AM/FinalAssociation_Pur_Third}
  \includegraphics[width=0.325\textwidth]{AM/FinalAssociation_Pro_Third}
  \caption[Effciency, purity and their product for the three options for the final association]{The efficiency (left hand plot), purity (middle plot) and the product of both (right hand plot) for the three options for the final association.\label{plot:AMWFFA}}
\end{figure}

A comparison of the three options of the purity and efficiency of this step only can be seen in Figure~\ref{plot:AMWFFA}. It is clearly visible that assigning the track always to the first vertex leads to the highest efficiency but also to the worst purity. Furthermore, there is only a small difference between the two different search options. Therefore, no difference in the distributions for efficiency and purity can be seen. Associating the track to the closest vertex along the z has been chosen as the default option. 


\subsection{Defining the quality of the association\label{sec:AMWFQD}}

As already noted earlier, an integer value is stored additionally to the created association. This value is designed to provide the information about the quality of the association only in terms of purity. For example, the highest quality stands for a purity greater than $90\,\%$.  The classification into these quality categories is based on the minimum distance from the track to the vertex, the step (1,2 or 3) and the iteration at which the association has been created. 

\subsubsection{First iteration}
In the first iteration the track can be associated in all three presented steps. Associations created in this iteration can obtain a quality between six (very high purity) and three. Smaller qualities are reserved for possible additional iterations. In Figure~\ref{plot:AMWFQualityI1}, the cumulative purity is shown as a function of the distance between the track and the associated vertex. There are actually five different distributions that need to be examined. \\
As an example the categorization is demonstrated for the association using the track weight. As it can be seen for a distance smaller than 0.004\cm the purity is above $90\,\%$. Hence, these association (which are about $28\,\%$) obtain a quality of 6. All other associations created at this step obtain a quality of 5 since the purity stays above $70\,\%$. \\
Next one is for the association based on the information of the secondary vertices. Here, it can be seen that for a distance smaller than 8.0\cm the purity is above $70\,\%$ (about $30\,\%$ of the tracks) which leads to a quality of 5. Then, the purity is between $50\,\%$ and $70\,\%$ for all distances. For these associations the quality is set to 4. \\
The other three quality value distributions stand for the three different types of the final association. The particular cut values for the categorization can be found in Table~\ref{tab:AMWFQDI1}. Here, it needs to be said that the cut value of 40\cm for the third step using the closest vertex in three dimension and only along the z axis is considered as a landmark.

\begin{figure}[!ht]
  \centering
  \includegraphics[width=0.45\textwidth]{AM/QualityAnalyzer_0_1D}
  \includegraphics[width=0.45\textwidth]{AM/QualityAnalyzer_0_1D_NonCum}
  \caption[Cumulative and normal purity of iteration one vs distance to define quality]{The cumulative (left hand plot) and normal (right hand plot) purity for the five different types of associations at the first iteration as as function of the distance between track and associated vertex.\label{plot:AMWFQualityI1}}
\end{figure}

%%
%% This is just the table of curt values for the quality definition for iteration one
%%
%% Iteration | Q and p | step | cv1 | cv2 | cv3
%%

\begin{table}[h]
\begin{center}
\caption{The cut values in \cm for the distance between track and associated primary vertex for the different steps in iteration one to define the quality of the association. This quality represents a certain purity.}
\label{tab:AMWFQDI1}
\begin{tabular}{c c | c | c | c | c}

 \textbf{Quality q} & & 6 & 5 & 4 & 3 \\
 cumulative purity & & $>90\,\%$ & $>70\,\%$ & $>50\,\%$ &  $<50\,\%$ \\
\hline \hline
Step 1 & & $ \leq0.004 $ & $ >0.004 $ & $ $ & $$ \\
\hline
Step 2 & &  & $ \leq8. $ & $ >8. $ & $$ \\
\hline
\multirow{3}{*}{Step 3} 
                        & 1st & $ $ & $\leq0.03$ & $\leq0.2$ & $>0.2$ \\
\cline{2-6}
                        & 3D & $ $ & $\leq0.03$ &  $\leq40.$ & $>40.$ \\
\cline{2-6}
                        & z &    $ $ & $\leq0.03$ &  $\leq40.$ & $>40.$ \\
\end{tabular}
\end{center}
\end{table}

\subsubsection{Second iteration}

In this iteration the association can only be created in the third step as explained in Section~\ref{sec:AMWorkflow}. In this iteration only the quality classes two and one are possible. The classification of the association created in this second iteration also depends on the step at which the track has been association in the first iteration and the distance of this first association. The idea behind this is that for small distances of the first iteration it is likely that the track could also come from another primary vertex. To account for that tracks which have a very small distance in the first iteration and a rather small distance at the second obtain a better quality for the second association than the rest. For this, the cut value of the distance of the association is set to that value as soon as the bin-by-bin purity shown in the right hand plot of Figure~\ref{plot:AMWFQualityI1} reaches a kind of a plateau. Only the purity distribution of the first step does not seem to flatten. Here, the value is chosen at which the purity falls below $50\,\%$.

\begin{figure}[!ht]
  \centering
  \includegraphics[width=0.3\textwidth]{AM/QualityAnalyzer_1_2D_FV_0}
  \includegraphics[width=0.3\textwidth]{AM/QualityAnalyzer_1_2D_3D_0}
  \includegraphics[width=0.3\textwidth]{AM/QualityAnalyzer_1_2D_Z_0}
  \caption[Cumulative purity of iteration two vs relative difference to define quality for smaller distances of association one]{The cumulative purity for always associating the track to the first vertex (left hand plot), looking for the closest in 3D (middle plot) and in z only (right hand plot) of the second iteration as a function of the distance of the second association. Here, the association of the first association has a rather small distance. \label{plot:AMWFQualityI20}}
\end{figure}

\begin{figure}[!ht]
  \centering
  \includegraphics[width=0.3\textwidth]{AM/QualityAnalyzer_1_2D_FV_1}
  \includegraphics[width=0.3\textwidth]{AM/QualityAnalyzer_1_2D_3D_1}
  \includegraphics[width=0.3\textwidth]{AM/QualityAnalyzer_1_2D_Z_1}
  \caption[Cumulative purity of iteration two vs relative difference to define quality for larger distances of association one]{The cumulative purity for always associating the track to the first vertex (left hand plot), looking for the closest in 3D (middle plot) and in z only (right hand plot) of the second iteration as a function of the distance of the second association. Here, the association of the first association has a rather larger distance. \label{plot:AMWFQualityI21}}
\end{figure}

First, the cut values for the first iteration need to be defined. As motivated by the right hand plot in Figure~\ref{plot:AMWFQualityI1} the cut value for the first step is set to 0.5\cm, for the second step it is set to 2.\cm. For the third steps the cut values are as follows: if associating the track to the first vertex always the value is set to 0.03\cm, if associating the track to the closest vertex in all three dimensions or in z only the value is set to 0.1\cm. These cut values can be found in Table~\ref{tab:AMWFQDI1nc}.\\
Next, the three different options have to be analyzed separately.

%%
%% This is just the table of curt values for the quality definition for iteration one
%%
%% Iteration | Q and p | step | cv1 | cv2 | cv3
%%

\begin{table}[h]
\begin{center}
\caption[Cut values for the distance between track and vertex to divide the association into two qulies on which the qualities of association created in iteration 2 are defined.]{The cut values in cm for the distance between track and associated primary vertex for the different steps in iteration 1. Tracks with association 1 with a smaller distance are categorized in the next iteration into quality class 1, all others in class 2. An exception are tracks which are associated using secondary vertices in iteration 1. These tracks are always categorized into two quality classes depending on the distances of association 1 and 2.}
\label{tab:AMWFQDI1nc}
\begin{tabular}{c c c}

  & & cut value /\cm  \\
\midrule[2pt]
Step 1 & & $ \leq0.5 $ \\
\midrule
Step 2 & & $ \leq2. $ \\
\midrule
\multirow{3}{*}{Step 3} 
                        & 1st & $ \leq0.3 $ \\
\cmidrule{2-3}
                        & 3D & $ \leq0.1 $  \\
\cmidrule{2-3}
                        & z &    $ \leq0.1 $  \\
\end{tabular}
\end{center}
\end{table}

First, associating the track to the first vertex of the vertex list. As it can be seen on the left hand plot in Figure~\ref{plot:AMWFQualityI21} if the track has been associated in the first step of the first iteration a cut value of 0.05\cm can be motivated to achive a purity greater than $30\,\%$. Hence, all association obtain a quality of 2. All associations with greater values obtain a quality of 1. If the track has been associated in the second step of iteration one the cut value is 0.2\cm for tracks with a small first distance and 0.7\cm for tracks with a greater distance to the associated vertex in the first iteration. All associations with a distance of the second iteration of smaller than the particular cut values obtain a quality of 2, all others a quality of 1. The purity of the second iteration for tracks that have been associated in the third step of iteration one is not measurable. In this case, no cut value is set and all associations obtain a quality of 1.\\ 
Second and third, the cut values for the other two techniques are the same. If the track has been associated at the first step of iteration one the cut value is set to 0.05\cm for the second iteration. If the track has been associated at iteration one in the second step the cut value is set to 0.3\cm and 1.0\cm, respectively. If the track has been associated in the third step at iteration one the cut value for the second iteration is set to 0.04\cm. \\ 
As it can be seen in Figure~\ref{plot:AMWFQualityI21} the cumulative purity of the second associations for tracks which have a rather large distance at the first associations drops soon below $30\,\%$. Except for tracks which have been associated in the first iteration using secondary vertices. Here, the purity stays above $30\,\%$ for a given range. \\
All cut values are summarized in Table~\ref{tab:AMWFQDI2}.

%%
%% This is just the table of cur values for the quality definition for iteration two
%%
%% Iteration | Q and p | step | cv1 | cv2 | cv3
%%

\begin{table}[h]
\begin{center}
\caption[Cut values for the distance between track and vertex to define the qualities of the association created in iteration 2.]{The cut values in cm for the distance between track and associated primary vertex for the different steps in iteration 2 to define the quality of the association. This quality represents a  cumulative purity of smaller or greater than $30\,\%$. The given number for the step represents the step at which the association is created in iteration 1. This categorization is only applied on tracks whose first association fulfill the criteria given in Table~\ref{tab:AMWFQDI1nc}. Exceptions are tracks which are associated in step 2 of iteration 1. For those, the cut values given in brackets are used.}
\label{tab:AMWFQDI2}

\begin{tabular}{l c c c c}
 	& \textbf{Quality q} & & \textbf{2} & \textbf{1}  \\
 	& cumulative purity & & $\geq30\,\%$ & $<30\,\%$    \\
\midrule[2pt]
\multirow{3}{*}{Always first vertex} 
	& Step 1 & & $\leq0.05$ & $>0.05$ \\
\cmidrule{2-5} 
	 & Step 2 & & $\leq0.2 \left(\leq0.7\right)$ & $>0.2\left(>0.7\right)$ \\
\cmidrule{2-5} 
	& Step 3 & &  &  \\
\midrule[2pt]
\multirow{3}{*}{Closest 3D} 
	& Step 1 & & $\leq0.05$ & $>0.05$ \\
\cmidrule{2-5} 
	 &  Step 2 & & $\leq0.3 \left(\leq1.\right)$ & $>0.3\left(>1.\right)$ \\
\cmidrule{2-5} 
	& Step 3 & & $\leq0.04$ & $>0.04$ \\
\midrule[2pt]
\multirow{3}{*}{Closest in z} 
	& Step 1 & & $\leq0.05$ & $>0.05$ \\
\cmidrule{2-5} 
	 &  Step 2 & & $\leq0.3 \left(\leq1.\right)$ & $>0.3\left(>1.\right)$ \\
\cmidrule{2-5} 
	& Step 3 & & $\leq0.04$ & $>0.04$ \\
\end{tabular}

\end{center}
\end{table}

\subsubsection{Third and following iterations}

It can be seen in Figure~\ref{plot:AMWFQualityI30} that the purity is already very low for the third iteration. Therefore, all associations created at the third or following iterations obtain a quality of 0. 

\begin{figure}[!ht]
  \centering
  \includegraphics[width=0.45\textwidth]{AM/QualityAnalyzer_2_2D_3D_0}
  \includegraphics[width=0.45\textwidth]{AM/QualityAnalyzer_2_2D_Z_0}
  \caption[Cumulative purity of iteration three vs relative difference to define quality for smaller distances of association one]{The cumulative purity of the third iteration as a function of the distance of the third association. Here, the association of the first association has a rather small distance. Shown are only the search for the closest vertex in three dimension (left hand plot) and only along the z-axis (right hand plot).\label{plot:AMWFQualityI30}}
\end{figure}

In Figure~\ref{plot:AMWFQualityDist} the distribution of the quality classes are shown. Associating the track to the closest vertex in all three dimensions and only along the z-axis lead to indistinguishable results. As it can be seen for the first iteration most of the associations obtain a quality of five (around $60\,\%$), only of the track is associated to the first vertex always a quality of three can be obtained in an observable fraction. In the second iteration about $5\,\%$ and $1\,\%$ of the association get a quality of two, respectively. The fact that the numbers of associations from the second (quality one and two) and iteration three (quality zero) reach not the number of the first iteration comes from the fact that events are possible with only one or two reconstructed vertices.  

\begin{figure}[!ht]
  \centering
  \includegraphics[width=0.55\textwidth]{AM/AMValidator_Quality}
  \caption[Distribution of the quality classes]{The distribution of the quality classes. The integral has been scaled in such way that the content of the qualities from 3 to 6 sum up to 1. Due to the fact that not always two or more primary vertices are reconstructed the integral of quality 1 and 2 as well as of quality 0 do not sum up to 1. Because of the similar results for looking for the closest vertex along the z-axis only or in all three dimensions the two distributions overlay in this plot. \label{plot:AMWFQualityDist}}
\end{figure}

\section{Output\label{sec:AMOutput}}

The created output are two maps each containing different directions for the association. Either a map containing a list of vertices each one with a list of assigned tracks and their quality is created or the other way around. Then, a list of tracks is created for each vertex with a list of associated primary vertices and the corresponding quality. The first map contains a sorted list of vertices. For this, the technique described in Section~\ref{sec:IntroSigVer} is again applied but for all tracks associated to the particular vertex and not only based on the track which were used to fit the vertex. Hence, a different order of the primary vertices is possible.\\
in Figure~\ref{plot:AMOutput} a sketch is shown of the two possible output maps for a very simple case is shown. Imagine a very simple case with only three reconstructed tracks and two primary vertices. This is of course not possible but just for illustration. After one association track one and three are associated to the first primary vertex while track two is associated to the second primary vertex. After a possible second iteration for track one and three also the second primary vertex is associated and for track two also the first primary vertex. Connected with each association is a quality by which the number of the iteration at which the association was made can be identified. On the left hand side it is shown for tracks being associated to vertices on the right hand side its the other way around.

\begin{figure}[!ht]
  \centering
  \includegraphics[width=0.55\textwidth]{AM/Association}
  \caption[Sketch of the two different output maps]{A sketch of the two possible output maps. On the left hand side tracks are associated to vertices, on the right hand side its the other way around. Solid arrows stand for the first association, dashed arrows for a possible second association. Connected to each arrow is a association quality by which the number of the association can be identified. \label{plot:AMOutput}}
\end{figure}
