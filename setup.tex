
\input{general/command-definitions}

%%%%%%%%%%%%%%%  Title page %%%%%%%%%%%%%%%%%%%%%%%%
\documentclass[11pt,twoside, openright, a4paper, pdftex, tdr]{general/new-cms-tdr}
\usepackage[left=29mm,right=25mm,top=25mm,bottom=14mm,includeheadfoot]{geometry}%includehead

\usepackage{multirow}

\newboolean{draft}
\setboolean{draft}{false} %%set comments and lipsums on and off 

\newboolean{publication}
\setboolean{publication}{true} %%include acknowledgements and different title page in publication version

\newcommand{\chooseTitlepage}{\ifthenelse{\boolean{publication}} {\input{title}} {\input{title_submission}} }
\newcommand{\includeAcknowledgement}{\ifthenelse{\boolean{publication}} {\input{acknowledgement}} {} }


\usepackage{xcolor}
\definecolor{lightblue}{rgb}{0.85,0.85,0.92}
\definecolor{gray}{gray}{0.6}
\usepackage{lipsum}
\newcommand{\lorem}{ \ifthenelse{\boolean{draft}} {\textcolor{lightblue}{\lipsum}} {} }
\setlipsumdefault{1-2}

\renewcommand{\arraystretch}{1.1}

%define comments
\newcommand{\comment}[1]{ \ifthenelse{\boolean{draft}} {\textcolor{red}{#1}} {} }
%\usepackage{ulem}
%\newcommand{\comment}[1]{ \ifthenelse{\boolean{draft}} {\textcolor{red}{\sout{#1}}} {} }

%set tau note include path
\newcommand{\notepath}{../tauNote}
%set figure include path
\newcommand{\notefigurepath}{\notepath/figures/}
%\input{\notepath/definitions}

%\renewcommand{\pairwidth}{.481\textwidth}

\usepackage[utf8,latin1]{inputenc}
\usepackage{upgreek}
\usepackage[german,english]{babel}
\usepackage{tikz}
%\input{\notepath/tikzSetup}
%\input{tikzSetupAddon}
\usepackage[width=0.9\textwidth, skip=8pt, format=plain, labelfont=bf]{caption}
\KOMAoption{captions}{tableheading, bottombeside}

\fancypagestyle{mystyle}{% plain neu definieren
  \fancyhf{} % reset
  \fancyfoot[C]{\thepage} % Seite zentriert
  \renewcommand{\headrulewidth}{0pt} %Kopflinie löschen
  \renewcommand{\footrulewidth}{0pt}% Fußlinie löschen
}

\makeatletter
\renewcommand{\fps@figure}{htbp}
\renewcommand{\fps@table}{htbp}
\makeatother

\usepackage{lettrine}
\usepackage{booktabs}
\usepackage{color}
\definecolor{RWTHblue}{RGB}{0,84,159}%RWTH blau
\definecolor{RWTHlightblue}{RGB}{142,186,229}%RWTH hellblau
\usepackage[%
colorlinks, % verwende farbige Links
linkcolor=black, % Linkfarbe ist RWTH blau
citecolor=RWTHlightblue, % Zitatfarbe ist RWTH blau
bookmarks, % erstelle Bookmarks der Links
bookmarksopen, % Bookmarks werden beim Öffnen des Dokumentes ebenfalls geöffnet
bookmarksopenlevel=2,
urlcolor=RWTHblue, % Hyperlinks sind RWTH blau 
bookmarksnumbered, % Bookmarks sind nummeriert
pdfborder=0,
plainpages=false,
pdfpagelabels,
final % Endversion 
]{hyperref}
\usepackage[numbers,sort&compress]{natbib}
